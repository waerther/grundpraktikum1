\section{Diskussion}
\label{sec:Diskussion}

Im Folgenden sind Faktoren genannt, die die Replizierbarkeit des oben behandelten Experimentes beeinflussen können.
Darunter ist unter anderem der kleine „systematische Fehler“, dass der Deckel die jeweiligen Reservoirs nicht
zu hundert Prozent von außen abdichten können, da diese nur auf dem Gefäß aufliegen.
Weiterhin sind die Skalen des Manometers nicht ganz so eindeutig wie vergleichsweise das Thermometer für die Wassertemperatur.
Außerdem kommt die Zeitunsicherheit der Messung hinzu.
Es wurde mit einer Stoppuhr im Minutentakt die Messwerte notiert, jedoch ließt man dabei die verschiedenen Werte
um einige wenige Sekunden zu unterschiedlichen Zeiten ab.

Auffallend ist, dass die Güteziffer stark von der idealen Güteziffer abweicht. Ein Grund für diese
Abweichung ist die Irreversibilität des Prozesses, obwohl bei der Berechnung der Güteziffer von
einem reversiblen Prozess ausgegangen wurde. Außerdem kommt es durch Reibung zu Energieverlusten.
Es wird weiter davon ausgegangen, dass der Kompressor adiabatisch komprimiert, was aufgrund natürlicher
Umstände nicht möglich ist.

All das hat Einflüsse aufdie Messung und somit auch auf die daraus resultierenden Werte, 
wie die Güteziffer $\nu$ und Kondensationswärme $L$, den Massendurchsatz, und schließlich die mech. Kompressionsleistung $N_\text{mech}$.