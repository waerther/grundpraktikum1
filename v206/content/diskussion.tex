\section{Diskussion}
\label{sec:Diskussion}

Im folgenden sind Dinge genannt, die die Replizierbarkeit des oben behandelten Experimentes beeinflussen können.
Darunter ist unter anderem der kleine „systematische Fehler“, dass der Deckel die jeweiligen Reservoirs nicht
zu hundert Prozent von außen abdichten können, da diese nur auf dem Gefäß aufliegen.
Weiterhin sind die Skalen des Manometers nicht ganz so eindeutig wie vergleichsweise das Thermometer für die Wassertemperatur.
Außerdem kommt die Zeitunsicherheit der Messung hinzu.
Es wurde mit einer Stoppuhr im Minutentakt die Messwerte notiert, jedoch ließt man dabei die verschiedenen Werte
um einige wenige Sekunden zu unterschiedlichen Zeiten ab.

All das hat Einflüsse auf verschiedene Messwerte, wie zum Beispiel dem Druck (durch die Deckelisolierung) und
somit auch auf die daraus resultierenden Werten, wie die mech. Kompressionsleistung $N_\text{mech}$.
Die anderen Werte, also Massendurchsatz, Kondensationswärme $L$ und die Güteziffer $\nu$.