\section{Auswertung}
\label{sec:Auswertung}

\subsection{Bestimmung der Güteziffer}

\subsubsection{Aufg. a: Messdaten und Diagramm}

Die gemessenen Daten werden tabellarisch dargestellt. Alle Größen wurden in SI-Einheiten umgerechnet, die Temperatur zudem invers dargestellt,
die Zeit für die Ausgleichsrechnung quartiert und gemäß der Versuchsanleitung \cite{v206} den Drücken $\textit{p}_\textit{a}$ und $\textit{p}_\textit{b}$ 1 bar bzw. 100000 Pa addiert:

%\begin{landscape}
  \begin{table}
    %\centering
    \caption{Messdaten und berechnete Werte}
    \label{tab:some_data}
    \sisetup{table-format=1.2}
    %\resizebox{.5/textwidth}{!}{%
    \begin{tabular}{S[table-format=3.0] S S S S S S S S S S S S[table-format=3.2]}
      \toprule
      %{$f$} & {$l_\text{start}$} & {$l_1$} & {$l_{\text{kor},1}$} & {$B_1$} \\
      {$t (\unit{\second})$} &
      {$t^{2} (\unit{\second\squared})$}&
      {$T_{1} (\unit{\celsius})$}&
      {$T_{2} (\unit{\celsius})$}&
      {$T_{1} (\unit{\kelvin})$}&
      {$T_{2} (\unit{\kelvin})$}&
      {$p_{a} (\unit{\bar})$}&
      {$p_{b} (\unit{\bar})$}&
      {$p_{a+} (\unit{\pascal})$}&
      {$p_{b+} (\unit{\pascal})$}&
      {$P (\unit{\watt})$}&
      {$1/T_{1} (1/\unit{\kelvin})$}& \\
      \midrule
      
      
      %    0  &         0   &   21,50   &    21,50   &  294,65   &  294,65   &      4,20   &      3,80   &  520000,00   &   480000,00   &        &      0,0034  \\
      %   60  &      3600   &   23,30   &    20,70   &  296,45   &  293,85   &      3,90   &      5,30   &  490000,00   &   630000,00   &        &      0,0034  \\
      %  120  &     14400   &   24,70   &    19,30   &  297,85   &  292,45   &      3,60   &      5,90   &  460000,00   &   690000,00   &        &      0,0034  \\
      %  180  &     32400   &   26,40   &    17,70   &  299,55   &  290,85   &      3,60   &      6,00   &  460000,00   &   700000,00   &        &      0,0033  \\
      %  240  &     57600   &   27,90   &    16,60   &  301,05   &  289,75   &      3,60   &      6,40   &  460000,00   &   740000,00   &        &      0,0033  \\
      %  300  &     90000   &   29,50   &    15,60   &  302,65   &  288,75   &      3,50   &      6,80   &  450000,00   &   780000,00   &  125,00    &      0,0033  \\
      %  360  &    129600   &   31,00   &    14,60   &  304,15   &  287,75   &      3,30   &      7,00   &  430000,00   &   800000,00   &  125,00    &      0,0033  \\
      %  420  &    176400   &   32,50   &    13,50   &  305,65   &  286,65   &      3,20   &      7,40   &  420000,00   &   840000,00   &  125,00    &      0,0033  \\
      %  480  &    230400   &   33,80   &    12,50   &  306,95   &  285,65   &      3,00   &      7,50   &  400000,00   &   850000,00   &  125,00    &      0,0033  \\
      %  540  &    291600   &   35,10   &    11,50   &  308,25   &  284,65   &      2,40   &      8,00   &  340000,00   &   900000,00   &  125,00    &      0,0032  \\
      %  600  &    360000   &   36,20   &    10,60   &  309,35   &  283,75   &      2,80   &      8,10   &  380000,00   &   910000,00   &  125,00    &      0,0032  \\
      %  660  &    435600   &   37,50   &     9,60   &  310,65   &  282,75   &      2,70   &      8,50   &  370000,00   &   950000,00   &  125,00    &      0,0032  \\
      %  720  &    518400   &   38,50   &     8,70   &  311,65   &  281,85   &      2,80   &      8,90   &  380000,00   &   990000,00   &  125,00    &      0,0032  \\
      %  780  &    608400   &   39,60   &     7,80   &  312,75   &  280,95   &      2,50   &      9,00   &  350000,00   &  1000000,00   &  125,00    &      0,0032  \\
      %  840  &    705600   &   40,60   &     7,10   &  313,75   &  280,25   &      2,40   &      9,10   &  340000,00   &  1010000,00   &  115,00    &      0,0032  \\
      %  900  &    810000   &   41,50   &     6,20   &  314,65   &  279,35   &      2,40   &      9,30   &  340000,00   &  1030000,00   &  115,00    &      0,0032  \\
      %  960  &    921600   &   42,40   &     5,50   &  315,55   &  278,65   &      2,20   &      9,70   &  320000,00   &  1070000,00   &  115,00    &      0,0032  \\
      % 1020  &   1040400   &   43,30   &     4,70   &  316,45   &  277,85   &      2,10   &      9,90   &  310000,00   &  1090000,00   &  115,00    &      0,0032  \\
      % 1080  &   1166400   &   44,10   &     4,00   &  317,25   &  277,15   &      2,10   &     10,10   &  310000,00   &  1110000,00   &  115,00    &      0,0032  \\
      % 1140  &   1299600   &   44,80   &     3,30   &  317,95   &  276,45   &      2,10   &     10,20   &  310000,00   &  1120000,00   &  115,00    &      0,0031  \\
      % 1200  &   1440000   &   45,60   &     2,70   &  318,75   &  275,85   &      2,00   &     10,50   &  300000,00   &  1150000,00   &  115,00    &      0,0031  \\
      % 1260  &   1587600   &   46,30   &     2,10   &  319,45   &  275,25   &      2,00   &     10,80   &  300000,00   &  1180000,00   &  115,00    &      0,0031  \\
      % 1320  &   1742400   &   47,10   &     1,40   &  320,25   &  274,55   &      1,90   &     11,00   &  290000,00   &  1200000,00   &  115,00    &      0,0031  \\
      % 1380  &   1904400   &   47,80   &     0,80   &  320,95   &  273,95   &      1,80   &     11,10   &  280000,00   &  1210000,00   &  115,00    &      0,0031  \\
      % 1440  &   2073600   &   48,40   &     0,30   &  321,55   &  273,45   &      1,80   &     11,20   &  280000,00   &  1220000,00   &  115,00    &      0,0031  \\
      % 1500  &   2250000   &   49,00   &    -0,20   &  322,15   &  272,95   &      1,80   &     11,60   &  280000,00   &  1260000,00   &  115,00    &      0,0031  \\
      % 1560  &   2433600   &   49,50   &    -0,70   &  322,65   &  272,45   &      1,80   &     11,80   &  280000,00   &  1280000,00   &  115,00    &      0,0031  \\
      % 1620  &   2624400   &   50,00   &    -0,10   &  323,15   &  273,05   &      1,80   &     12,00   &  280000,00   &  1300000,00   &  115,00    &      0,0031  \\
      \bottomrule
    \end{tabular}
    %}
  \end{table}
%\end{landscape}

%plot

\subsubsection{Aufg. b: Bestimmung der Parameter der Näherungsfunktion}

Eine nicht-lineare Ausgleichsrechnung der Temperaturverläufe mittels EXCEL mithilfe der Näherungsfunktion

\begin{equation} 
  T(t) = At^2 + Bt + C 
\end{equation}

ergibt die folgenden Parameter für $T_{1} (\unit{\kelvin})$:
\begin{equation} 
  T(t) = At^2 + Bt + C 
\end{equation}

% quadratische regression für T_1 und T_2

\subsubsection{Aufg. c: Bestimmung der Differentialquotienten}

% usepackage{}