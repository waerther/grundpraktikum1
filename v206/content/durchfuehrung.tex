\section{Durchführung}
\label{sec:Durchführung}

Im folgenden ist eine ergänzende Skizze aus der Versuchsanleitung beigefügt

\begin{figure}
    \centering
    \includegraphics[width=8cm] {pictures/Durchführung.png} \cite{v206} 
    \caption{Schematische Darstellung der kompletten Messaparatur}
    \label{fig:messaparatur}
\end{figure}

Bei der Durchführung sind einige Dinge zu beachten.
Zuerst wird die ganaue Menge an Wasser für beide Reservoiren mithilfe eines Messgefäßes gemessen.
Danach wird die exakte Temperatur des Wassers durch die beiden digitalen Thermometer ermittelt.
Ebenfalls wird jeder andere Wert bereits notiert(Druck, spezifische Wärmekapazität der Kupferschlange).
Für ein sinnvolles Ergebnis sind bei dem gesamten Experiment Rührmotoren einzusetzen,
damit das Wasser homogen aufgewärmt und abgekühlt wird. 
Das Einschalten des Kompressors $K$ einzuschalten leitet schließlich den 
Wärmetransport vom einen Reservoir in das nächste. \\
\\
Ab hier werden minütlich Werte notiert: Die Drücke $\text{p}_\text{a}$, 
$\text{p}_\text{b}$, die Temperaturen $T_{1}$, $T_{2}$ und die Leistung $\text{P}$ des Kompressors.