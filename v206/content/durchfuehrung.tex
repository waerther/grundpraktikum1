\section{Durchführung}
\label{sec:Durchführung}

Im folgenden ist eine ergänzende Skizze aus der Versuchsanleitung beigefügt

\begin{center}\includegraphics[width=8cm,height=8cm] {pictures/Durchführung.png} \cite{v206} \end{center}

Bei der Durchführung sind einige Dinge zu beachten.
Zuerst misst man die ganaue Menge Wasser in beiden Reservoiren ab.
Dies sind in unserem Fall 3 Liter pro Behälter.
Danach wird die exakte Temperatur des Wassers durch die beiden digitalen Thermometer ermittelt.
Ebenfalls wird jeder andere Wert bereits notiert(Druck, spezifische Wärmekapazität der Kupferschlange).
Für ein sinnvolles Ergebnis sind bei dem gesamten Experiment Rührmotoren einzusetzen,
damit das Wasser homogen aufgewärmt und abgekühlt wird. 
Nun ist man bereit den Kompressor $K$ einzuschalten und somit den 
Wärmetransport vom einen Reservoir ins nächste einzuleiten.

Ab hier müssen minütlich folgende Werte notiert werden: Die Drücke $\text{p}_\text{a}$, 
$\text{p}_\text{b}$, die Temperaturen $T_{1}$, $T_{2}$ und die Leistung $\text{P}$ des Kompressors.