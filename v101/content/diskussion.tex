\section{Diskussion}
\label{sec:Diskussion}

Die Messung des Eigenträgheitsmoment in Abbildung (2) lässt auf eine gute Messung schließen,
da hier die Abweichung der linearen Regression sehr gering ist.
Außerdem wird das Eigenträgheitsmoment als zu klein angenommen, als das es in den Rechnungen berücksichtigt werden müsste.
Dadurch vereinfachen sich die Rechnungen, jedoch wird dadurch der Fehler etwas größer.

Die Messung der Trägheitsmomente der Objekte wird dadurch erschwert, dass letzlich die Fähigkeit des Experementators die Zeit zur 
richtigen Zeit zu stoppen, gefordert ist.
Aufgrund der schnellen Umdrehungen, ist dies eine große experimentelle Schwierigkeit.
Dieser Fehler wird dadurch verringert, indem viele Messungen durchgeführt werden.

Des Weiteren sind die Körperteile stark vereinfacht angenommen und einige werden schlicht nicht berücksichtigt.
Beispielsweise werden die Hände, Füße aber auch die Gelenke der Figur vernachlässigt, so auch die Reibung.
Diese tritt in Form von Luftreibung an der Puppe selbst auf, aber vor allem in der Drillachse.
Dies sorgt dafür, dass sich die Amplituden der Umdrehungen drastisch verringern und die Messung ebenfalls erschweren, da die
Messung \enquote{per Augenmaß} deutlich erschwert wird.

\nocite{matplotlib}
\nocite{numpy}