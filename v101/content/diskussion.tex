\section{Diskussion}
\label{sec:Diskussion}

Die Messung des Eigenträgheitsmoment in Abbildung 2 lässt auf eine gute Messung schließen,
da hier die Abweichung der linearen Regression sehr gering ist.
Das Eigenträgheitsmoment ist $I_{D} =  (0.0016 \pm 0.0005) \, \unit{\kilo\gram\meter\squared}$.
Es ist also zu klein, als das es in den Rechnungen berücksichtigt werden müsste.
Dadurch vereinfachen sich die Rechnungen, jedoch wird dadurch der Fehler theoretisch etwas größer.

Die Messung der Trägheitsmomente der Objekte wird dadurch erschwert, dass letzlich die Fähigkeit des Experimentators die Zeit zur 
richtigen Zeit zu stoppen, gefordert ist.
Aufgrund der schnellen Umdrehungen, ist dies eine große experimentelle Schwierigkeit.
Dieser Fehler wird dadurch verringert, indem viele Messungen durchgeführt werden.
Es ergaben sich die Werte $I_{\text{Zyl}} = (3,056 \pm 1,475) \cdot 10^{-4} \, \unit{\kilo\gram\meter\squared}$
und $I_{\text{Zyl, theo}} = 4,4425 \cdot 10^{-4} \, \unit{\kilo\gram\meter\squared}$ für den Zylinder. Es gibt also eine Abweichung
von circa 31,21\% zum Theoriewert.
Bei der Kugel ergaben sich die Werte $ I_{\text{K, theo}} = 2.5128 \cdot 10^{-3} \, \unit{\kilo\gram\meter\squared}$ und
$ I_{\text{Kug}} = (1.8048 \pm 0.8413) \cdot 10^{-3} \, \unit{\kilo\gram\meter\squared}$. Zum Theoriewert ist also eine Abweichung von
28.32 \% zu erkennen.
Bei der Puppe traten zudem noch andere Probleme auf.
Die Körperteile werden stark vereinfacht und einige werden schlicht nicht berücksichtigt.
Beispielsweise werden die Hände, Füße aber auch die Gelenke der Figur vernachlässigt, so auch die Reibung.
Diese tritt in Form von Luftreibung an der Puppe selbst auf, aber vor allem in der Drillachse.
Dies sorgt dafür, dass sich die Amplituden der Umdrehungen drastisch verringern und die Messung ebenfalls erschweren, da die
Messung \enquote{per Augenmaß} deutlich erschwert wird.
Es ergaben sich für die \enquote{90°-Pose} die Werte $I_E =( 0,39985 \pm 0,07382) \cdot 10^{-3}  \unit{\kilo\gram\meter\squared}$ und
$I_{\text{theo}} = (0,477969 \pm 0,00495)  \unit{\kilo\gram\meter\squared}$. Diie Abweichung berechnet sich zu 16.34 \%,
welches ein zufriedenstellendes Ergebnis ist.
Bei der \enquote{T-Pose} ergaben sich die Werte $I_{\text{theo}} = (0,13192 \pm 0,00672) \cdot 10^{-3} \, \unit{\kilo\gram\meter\squared}$
und  $I_E = (0,24452 \pm 0,00524)  \cdot 10^{-3} \, \unit{\kilo\gram\meter\squared}$.
Daraus berechnet man eine Abweichung von 85 \% zum Theoriewert.
Abschließend wird noch das Verhältnis der Trägheitsmomente der Puppe angegeben mit
\begin{equation*}
    \frac{I_\text{Pose 1}}{I_\text{Pose 2}} = \frac{0,39985}{0,24452} = 1,63524 \text{ .}
\end{equation*}

\nocite{matplotlib}
\nocite{numpy}