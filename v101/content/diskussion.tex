\section{Diskussion}
\label{sec:Diskussion}

Die Messung des Eigenträgheitsmoment in Abbildung (2) lässt auf eine gute Messung schließen,
da hier die Abweichung von der lin. Regression sehr gering ist.
Außerdem wird das Eigenträgheitsmoment als zu klein angenommen, als das es in den Rechnungen berücksichtigt werden müsste.
Dadurch vereinfachen sich die Rechnungen, jedoch wird dadurch der Fehler etwas größer.

Die Messung der Trägheitsmomente der Objekte wird dadurch erschwert, dass letzlich die Fähigkeit des Experementators die Zeit zur 
richtigen Zeit zu stoppen, gefordert war.
Dadurch, dass die Umdrehungen teils sehr schnell waren, ist dies eine große experimentelle Schwierigkeit.
Dieser Fehler wird dadurch verringert, indem viele Messungen durchgeführt werden.

Desweiteren sind die Körperteile stark vereinfacht angenommen und einige wurden schlicht nicht berücksichtigt.
Beispielsweise sind die Hände, Füße aber auch die Gelenke der Figur vernachlässigt worden.
Es ist klar, dass das Ergebnis so nicht exakt sein kann.
Ebenfalls wird die Reibung vernachlässigt.
Diese trat in Form von Luftreibung an der Puppe selbst auf, aber vor allem in der Drillachse.
Dies sorgte dafür, dass sich die Aplituden der Umdrehungen drastisch verringerten und die Messung ebenfalls erschwerten, da die
Messung \enquote{per Augenmaß} deutlich erschwert wurde.

\nocite{matplotlib}
\nocite{numpy}