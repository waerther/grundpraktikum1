\section{Fehlerrechnung}
\label{sec:Fehlerrechnung}


Im Folgenden wird die allgemeine Fehlerrechnung, und alle wichtigen Größen der entsprechenden Rechnung, erklärt.
Die wichtigsten Werte dabei sind der Mittelwert und die Standartabweichung.
Der arithmetische Mittelwert wird dabei durch die Gleichung
\begin{equation}
    \stackrel{_{-}}{x}_{\text{arithm}}  = \frac{1}{N} \sum_{i=0}^{n} x_i
\end{equation}
beschrieben.
Die Standartabweichung ist durch
\begin{equation}
    \sigma  = \sqrt{\frac{1}{N - 1 } \sum_{i=0}^{N} (x_i -  \stackrel{_{-}}{x}_{\text{arithm}})^2}
\end{equation}
gegeben. Dabei entspricht N der Anzahl an Werten und $x_i$ ist jeweils ein mit einem Fehler gemessener Wert.
%\begin{align}
%    \text{Mittelwert:} & \stackrel{_{-}}{x}_{\text{arithm}}  = \frac{1}{N} \sum_{i=0}^{n} x_i & \\
%    \text{Standartabweichung: } & \sigma  = \sqrt{\frac{1}{N - 1 } \sum_{i=0}^{N} (x_i -  \stackrel{_{-}}{x}_{\text{arithm}})^2}
%\end{align}


Es ergibt sich ebenfalls die statistische Messunischerheit

\begin{equation}
    \increment \stackrel{_{-}}{x}_{\text{arithm}} = \frac{\sigma}{\sqrt{N}} = 
    \sqrt{\frac{1}{N(N - 1)} \sum_{i=0}^{N} (x_i -  \stackrel{_{-}}{x}_{\text{arithm}})^2} \text{ .}
\end{equation}

Sind nun mehrere fehlerbehaftete Größen in der Messung, wird der sich fortpflanzende Fehler wichtig.
Es gilt die \textit{Gaußsche Fehlerfortplanzung}

\begin{equation} \label{fehler:gauß}
    \increment f(y_1 ,y_2 ,...,y_N ) = \sqrt{\left(\frac{\text{d} f}{\text{d} y_{1}} \increment y_{1}\right)^2
    + \left(\frac{\text{d} f}{\text{d} y_{2}} \increment y_{2}\right)^2 + ... + 
    \left(\frac{\text{d} f}{\text{d} y_{N}} \increment y_{N}\right)^2
    } \text{ .}
\end{equation}