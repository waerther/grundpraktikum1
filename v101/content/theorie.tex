\section{Theorie \tiny{\cite{sample}}}
\label{sec:Theorie}
\setlength{\parindent}{0pt}

Die Rotationsbewegung eines Körpers lässt sich durch die physikalische Größen \textit{Drehmoment} $M$, das 
\textit{Trägheitsmoment} $I$ und die \textit{Winkelbeschleunigung} $\dot{\omega}$ beschreiben.
Das Trägheitsmoment ist gewissermaßen ein Widerstand, den ein Körper der Bewegung entgegenbringt.
Die Formel für das Trägheitsmoment lautet



\begin{equation*}
    I = \sum_{i} r_{i}^2 \cdot m_{i} 
\end{equation*}

wobei $m_{i}$ Massepunkte in dem Objekt von Interesse beschreibt und $r_{i}^2$ den Abstand von der Rotationsachse.
Nun möchten wir allerdings nicht Massepunkte beschreiben, sondern ein kompaktes Objekt.
Dies leistet folgende Integration:
\begin{equation}
    I = \int r^2 \text{dm}
\end{equation}

Diese Gleichungen sind in dieser Form nur dann richtig, wenn die Drehachse auch die Schwerpunktachse ist.
Sobald dies nicht mehr gewährleistet ist, muss man zu dem Satz von Steiner übergehen.
Dabei ist $I_{s}$ das Trägheitsmoment des Körpers, wenn die Schwerpunktachse der Rotationsachse entspricht
und $a$ ist der Abstand zur Rotationsachse:

\begin{equation}
    I = I_{s} + m \cdot a^2
\end{equation}

Ebenfalls ist oft das \textit{Drehmoment} $M$ von Interesse. Dies berechnet mit dem Winkel $\varphi$ durch die Formel:
\begin{equation} \label{eq:drehmoment}
    \vec{M} = \vec{r} \times \vec{F} \implies \lvert M \rvert = \lvert F \rvert \lvert r \rvert \sin{\varphi}
\end{equation}

Bei schwingungsfähigen Systemen wie bei einer Feder, lässt sich $M$ auch mithilfe der \textit{Winkelrichtgröße} $D$ beschreiben:
\begin{equation}
    M = D \cdot \varphi
\end{equation}

Dabei gilt für $D$ die wichtige Beziehung:
\begin{equation} \label{eq:T}
    T = 2\pi \sqrt{\frac{I}{D}}
\end{equation}

durch umstellen erhält man ebenfalls folgende Gleichungen:
\begin{align} \label{eq:Tumgestellt}
    I & = \frac{T^2D}{4\pi^2} & D & = \frac{4\pi^2I}{T^2} 
\end{align}

Zu beachten ist jedoch dass (\ref{eq:T}) und (\ref{eq:Tumgestellt}) ihre Gültigkeit nur für kleine Auslenkungen $\varphi$ wahren.
Es gelten die Näherungen der Kleinwinkelnäherung.
Möchte man also D berechnen, gibt es zwei Methoden.

\subsection*{Statische Methode}
Bei der statischen Methode wird mit einem Newtonmeter die Kraft $F$ bestimmt, die bei einer Auslenkung $\varphi$ wirkt.
Dadurch kann man dann durch /(\ref{eq:drehmoment}) und (ref{eq:Tumgestellt}) D bestimmen.
Dabei ist natürlich darauf zu achten, das Newtonmeter in 90° zum Hebelarm zu halten.

\subsection*{Dynamische Methode}
Bei der dynamischen Methode wird die Beziehung (\ref{eq:Tumgestellt}) benutzt.
Diesmal wird jedoch die Gleichung für I benutzt.
Es wird also die Schwingungsdauer $T$ gemessen und mit bekanntem Trägheitsmoment $I$ wird  $D$ bestimmt
Es sei angemerkt, dass durch die Anwendung beider Methoden $D$ und $I$ gleichtzeitig bestimmt werden können.