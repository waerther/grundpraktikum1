\section{Theorie}
\label{sec:Theorie}
\setlength{\parindent}{0pt}

Die Rotationsbewegung eines Körpers lässt sich durch die physikalischen Größen \textit{Drehmoment} $M$, das 
\textit{Trägheitsmoment} $I$ und die \textit{Winkelbeschleunigung} $\dot{\omega}$ beschreiben.
Das Trägheitsmoment ist gewissermaßen ein Widerstand, den ein Körper der Bewegung entgegenbringt.
Die Formel für das Trägheitsmoment lautet
\begin{equation*}
    I = \sum_{i} r_{i}^2 \cdot m_{i} \text{ ,}
\end{equation*}

wobei $m_{i}$ Massepunkte in dem Objekt beschreibt und $r_{i}^2$ den senkrechten Abstand von der Rotationsachse.
Nun werden allerdings nicht einzelne Massepunkte betrachtet, sondern eine kontinuierliche Massenverteilung.
Dies ergibt die folgende Integration
\begin{equation}
    I = \int r^2 \dif{m} \text{.}
\end{equation}

Diese Gleichungen sind in dieser Form nur dann richtig, wenn die Drehachse auch die Schwerpunktachse ist.
Sobald dies nicht mehr gewährleistet ist, ist der \textit{Satz von Steiner} erforderlich.
Dabei ist $I_{s}$ das Trägheitsmoment des Körpers, wenn die Schwerpunktachse der Rotationsachse entspricht
und $a$ der parallele Abstand zur Rotationsachse ist. Es folgt

\begin{equation} \label{eq:SatzvSteiner}
    I = I_{s} + m \cdot a^2 \text{.}
\end{equation}

Durch Anwendung dieser Gleichungen, ergibt sich das Trägheitsmoment der Kugel
\begin{equation} \label{eq:AllgTreagKugel}
  I_{\text{K}} = \frac{2}{5} m r^2 \text{ .}
\end{equation}

Ebenfalls ist das \textit{Drehmoment} $M$ von Interesse, welches mit dem Winkel $\varphi$ zwischen dem Orts- und Kraftvektor die Formel
\begin{equation} \label{eq:drehmoment}
    \vec{M} = \vec{r} \times \vec{F} \implies \lvert M \rvert = \lvert F \rvert \lvert r \rvert \sin{\varphi}
\end{equation}

ergibt. Bei schwingungsfähigen Systemen wie bei einer Feder, lässt sich $M$ auch mithilfe der \textit{Winkelrichtgröße} $D$ beschreiben.
Es gilt
\begin{equation} \label{eq:drehmomentwinkel}
    M = D \cdot \varphi \text{ .}
\end{equation}

Mit (\ref{eq:drehmoment}) und (\ref{eq:drehmomentwinkel}) ergibt sich
\begin{equation}
D = \frac{\lvert F \rvert \lvert r \rvert \sin{\vartheta}}{\varphi} \stackrel{F,r > 0}{=} \frac{F r \sin{\vartheta}}{\varphi} 
\stackrel{\vartheta = \frac{\pi}{2}}{=} \frac{F \cdot r}{\varphi} \text{ .}
\end{equation}

Dabei gilt für $D$ die wichtige Beziehung
\begin{equation} \label{eq:T}
    T = 2\pi \sqrt{\frac{I}{D}} \text{.}
\end{equation}

Durch umstellen ergeben sich die Gleichungen 
%\begin{align} \label{eq:Tumgestellt}
%    I & = \frac{T^2D}{4\pi^2} \\
%    D & = \frac{4\pi^2I}{T^2} 
%\end{align}



\begin{equation}\label{eq:Tumgestellt1}
    I = \frac{T^2D}{4\pi^2}
\end{equation}
und 
\begin{equation}\label{eq:Tumgestellt2}
    D = \frac{4\pi^2I}{T^2} \text{ .}
\end{equation}

Zu beachten ist jedoch, dass (\ref{eq:T}), (\ref{eq:Tumgestellt1}) und (\ref{eq:Tumgestellt2}) ihre Gültigkeit nur für kleine Auslenkungen $\varphi$ wahren.
Es gelten die Näherungen der Kleinwinkelnäherung.
Zur Berechnung von D bieten sich dafür zwei Methoden an.


% Korrektur: In Durchführung verschieben?
\subsection{Statische Methode} \label{sec:statmethode}
Bei der statischen Methode wird mit einem Newtonmeter die Kraft $F$ bestimmt, die bei einer Auslenkung $\varphi$ wirkt.
Dadurch lässt sich durch (\ref{eq:drehmoment}) und (\ref{eq:Tumgestellt2}) D bestimmen.
Dabei ist darauf zu achten, dass das Newtonmeter im rechten Winkel zum Hebelarm zu halten ist.

\subsection{Dynamische Methode} \label{sec:dynmethode}
Bei der dynamischen Methode wird die Beziehung (\ref{eq:Tumgestellt1}) benutzt.
Es wird also die Schwingungsdauer $T$ gemessen und mit dem bekannten Trägheitsmoment $I$ die Winkelrichtgröße $D$ bestimmt.
Es sei angemerkt, dass durch die Anwendung beider Methoden $D$ und $I$ gleichzeitig bestimmt werden können.