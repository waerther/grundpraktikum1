\section{Auswertung}
\label{sec:Auswertung}

\subsection{Vorbereitung}\label{subsec:Vorbereitung}
Um die Messungen auswerten zu können, muss zunächst das Eigenträgheitsmoment $I_{D}$ und die Winkelrichtgröße $D$ berechnet werden.

Mit (\ref{eq:drehmoment}) und (\ref{eq:drehmomentwinkel}) ergibt sich:
\begin{equation}
D = \frac{\lvert F \rvert \lvert r \rvert \sin{\vartheta}}{\varphi} \stackrel{F,r > 0}{=} \frac{F r \sin{\vartheta}}{\varphi} 
\stackrel{\vartheta = \frac{\pi}{2}}{=} \frac{F \cdot r}{\varphi}
\end{equation}

In Tabelle (\ref{tab:Winkelrichtgröße}) sind die entsprechenden Messwerte und die berechneten Winkelrichtgrößen.

\begin{table}
\centering
\caption{Messdaten zu Winkelrichtgröße, $r = 20 \unit{\centi\meter}$}
\label{tab:Winkelrichtgröße}
\begin{tabular}{c c c}
  \toprule
  $\varphi / °$  &  $F / \unit\newton$ & $D / \unit{\newton\meter}$ \\
  \midrule
              20 &        0.022 &     0.012605 \\
              30 &        0.054 &     0.020626 \\
              40 &        0.077 &     0.022059 \\
              50 &        0.092 &     0.021085 \\
              60 &        0.120 &     0.022918 \\
              70 &        0.144 &     0.023573 \\
              80 &        0.162 &     0.023205 \\
              90 &        0.188 &     0.023937 \\
             100 &        0.190 &     0.021772 \\
             110 &        0.200 &     0.020835 \\
             120 &        0.230 &     0.021963 \\
  \bottomrule
\end{tabular}
\end{table}

Es ergibt sich der Mittelwert:
\begin{align*}
  D_{mittel} = D  & = 0.021325364366313267 \pm 0.002952928161464375 \unit{\newton\meter}\\
  \implies D &  = 0.021 \pm 0.00295 \unit{\newton\meter}
\end{align*}

Um das Eigenträgheitsmoment zu bestimmen wird nach (\ref{eq:Tumgestellt}) I berechnet.
Dafür wird die Umlaufzeit gemessen, wobei der Stab mit zwei bekannten Massen an den Seiten ausgelenkt wird.
Dabei ist der Radius genau bestimmt.
Die Daten sind in Tabelle (\ref{tab:Eigenträgheitsmoment}) aufgetragen:

\begin{table}
  \centering
  \caption{Messdaten zum Eigenträgheitsmoment $I_{D}$}
  \label{tab:Eigenträgheitsmoment}
  \begin{tabular}{c c c c}
    \toprule
    $r / \unit\meter$  &  $T / \unit\second$ & $r^2 / (\unit\meter^2)$  & $T^2 / (\unit\second)^2$\\
    \midrule
      0.050   & 2.750   & 0.003   &  7.562  \\
      0.075   & 3.100   & 0.006   &  9.610  \\
      0.100   & 3.800   & 0.010   & 14.440  \\
      0.125   & 4.100   & 0.016   & 16.810  \\
      0.150   & 4.750   & 0.022   & 22.562  \\
      0.175   & 5.300   & 0.031   & 28.090  \\
      0.200   & 5.800   & 0.040   & 33.640  \\
      0.225   & 6.600   & 0.051   & 43.560  \\
      0.250   & 7.150   & 0.062   & 51.123  \\
      0.275   & 7.800   & 0.076   & 60.840  \\
    \bottomrule
    \end{tabular}
\end{table}

Aus diesen Daten können wir jetzt das Eigenträgheitsmoment bestimmen.
Dafür brauchen wir zunächst noch die Formeln für einen Zylinder.
Dieses lässt sich mit rechnerisch bestimmen, hier sei nur die Formel angegeben:
\begin{equation*}
  I_{\text{Zyl}} = \frac{m \cdot R^2} {2}
\end{equation*}

Mit dem Satz von Steiner (\ref{eq:SatzvSteiner}) ergibt sich:
\begin{equation} \label{eq:IZyl}
  I_{\text{Zyl}} = I_{D} + m_{\text{Zyl}}  \left( \frac{3 r^2 + h^2}{6} \right)
\end{equation}

Setzt man nun $I_{\text{Zyl}}$ in Gleichtung (\ref{eq:Tumgestellt}) ein und stellt um, erhält man folgende Gleichung:

\begin{equation}
  T^2 = \frac{4 \pi^2}{D} \cdot \left( ma^2 + I_{\text{Zyl}} \right) \stackrel{(\ref{eq:IZyl})}{=} 
  \frac{4 \pi^2}{D} \left( ma^2 + I_{D} + m_{\text{Zyl}}  \left( \frac{3 r^2 + h^2}{6} \right) \right)
\end{equation}

Nun erhält man durch lineare Regression eine Ausgleichsgerade der Form $y = a \cdot x + b$ mit den Parametern
$a = 724.885 ± 10.115$ und $b = 5.945 ± 0.400$. Damit ergibt sich folgender Plot:
\begin{figure}
  \caption{Regressionsgerade des Eigenträgheitsmomentes}
  \centering
  \includegraphics{pictures/Lineare Regression.pdf}
  \label{fig:LinReg}
\end{figure}