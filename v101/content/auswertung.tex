\section{Auswertung}
\label{sec:Auswertung}

\subsection{Vorbereitung}\label{subsec:Vorbereitung}
Um die Messungen auswerten zu können, muss zunächst das Eigenträgheitsmoment $I_{D}$ und die Winkelrichtgröße $D$ berechnet werden.\\
\\
Mit (\ref{eq:drehmoment}) und (\ref{eq:drehmomentwinkel}) ergibt sich:
\begin{equation}
D = \frac{\lvert F \rvert \lvert r \rvert \sin{\vartheta}}{\varphi} \stackrel{F,r > 0}{=} \frac{F r \sin{\vartheta}}{\varphi} 
\stackrel{\vartheta = \frac{\pi}{2}}{=} \frac{F \cdot r}{\varphi}
\end{equation}

In Tabelle (\ref{tab:Winkelrichtgröße}) sind die entsprechenden Messwerte und die berechneten Winkelrichtgrößen.

\begin{table}
\centering
\caption{Messdaten zu Winkelrichtgröße, $r = 20 \unit{\centi\meter}$}
\label{tab:Winkelrichtgröße}
\begin{tabular}{c c c}
  \toprule
  $\varphi / °$  &  $F / \unit\newton$ & $D / \unit{\newton\meter}$ \\
  \midrule
              20 &        0.022 &     0.012605 \\
              30 &        0.054 &     0.020626 \\
              40 &        0.077 &     0.022059 \\
              50 &        0.092 &     0.021085 \\
              60 &        0.120 &     0.022918 \\
              70 &        0.144 &     0.023573 \\
              80 &        0.162 &     0.023205 \\
              90 &        0.188 &     0.023937 \\
             100 &        0.190 &     0.021772 \\
             110 &        0.200 &     0.020835 \\
             120 &        0.230 &     0.021963 \\
  \bottomrule
\end{tabular}
\end{table}

Es ergibt sich der Mittelwert:
\begin{align*}
  D_{mittel} = D  & = (0.021325364366313267 \pm 0.002952928161464375) \, \unit{\newton\meter}\\
  \implies D &  = (0.021 \pm 0.00295) \, \unit{\newton\meter}
\end{align*}

Um das Eigenträgheitsmoment zu bestimmen wird $I$ nach (\ref{eq:Tumgestellt}) berechnet.
Dafür wird die Umlaufzeit gemessen, wobei der Stab mit zwei bekannten Massen an den Seiten ausgelenkt wird.
Dabei ist der Radius genau bestimmt.
Die Daten sind in Tabelle (\ref{tab:Eigenträgheitsmoment}) aufgetragen:

\begin{table}
  \centering
  \caption{Messdaten zum Eigenträgheitsmoment $I_{D}$}
  \label{tab:Eigenträgheitsmoment}
  \begin{tabular}{c c c c}
    \toprule
    $r / \unit\meter$  &  $T / \unit\second$ & $r^2 / (\unit\meter^2)$  & $T^2 / (\unit\second)^2$\\
    \midrule
      0.050   & 2.750   & 0.003   &  7.562  \\
      0.075   & 3.100   & 0.006   &  9.610  \\
      0.100   & 3.800   & 0.010   & 14.440  \\
      0.125   & 4.100   & 0.016   & 16.810  \\
      0.150   & 4.750   & 0.022   & 22.562  \\
      0.175   & 5.300   & 0.031   & 28.090  \\
      0.200   & 5.800   & 0.040   & 33.640  \\
      0.225   & 6.600   & 0.051   & 43.560  \\
      0.250   & 7.150   & 0.062   & 51.123  \\
      0.275   & 7.800   & 0.076   & 60.840  \\
    \bottomrule
    \end{tabular}
\end{table}

Aus diesen Daten können wir jetzt das Eigenträgheitsmoment bestimmen.
Aufgrund der Form der verwendeteten Gewichte wird zunächst noch die Formel für einen Zylinder benötigt,
welche sich rechnerisch bestimmen lässt:
\begin{equation*}
  I_{\text{Zyl}} = \frac{m \cdot R^2} {2}
\end{equation*}

Mit dem Satz von Steiner (\ref{eq:SatzvSteiner}) ergibt sich:
\begin{equation} \label{eq:IZyl}
  I_{\text{Zyl}} = I_{D} + m_{\text{Zyl}}  \left( \frac{3 r^2 + h^2}{6} \right)
\end{equation}

Nach Einsetzen von $I_{\text{Zyl}}$ in Gleichung (\ref{eq:Tumgestellt}) und Umstellen, ergibt sich:

\begin{equation}
  T^2 = \frac{4 \pi^2}{D} \cdot \left( 2ma^2 + I_{\text{Zyl}} \right) \stackrel{(\ref{eq:IZyl})}{=} 
  \frac{4 \pi^2}{D} \left( ma^2 + I_{D} + m_{\text{Zyl}}  \left( \frac{3 r^2 + h^2}{6} \right) \right)
\end{equation}

Durch lineare Regression mithilfe der Ausgleichsgerade 
\begin{equation*}
  y = a \cdot x + b 
\end{equation*}

mit den Parametern $a = 724.885 ± 10.115$ und $b = 5.945 ± 0.400$ ergibt sich der Plot in Abbildung (\ref{fig:Lineareregression}).

\begin{figure}[H]
  \centering
  \includegraphics{pictures/Lineare Regression.pdf}
  \caption{Regressionsgerade des Eigenträgheitsmomentes}
  \label{fig:Lineareregression}
\end{figure}

Nun wird durch Koeffizientenvergleich a und b in (\ref{eq:IZyl}) erkannt und eingesetzt.
Es ergibt sich:
\begin{align} \label{var:aundb}
  a & = \frac{8 \pi^2 m}{D} & b & = \frac{4 \pi^2}{D} \left(I_{D} + m_{\text{Zyl}} \frac{3 r^2 + h^2}{6} \right)
\end{align}

Wird (\ref{eq:IZyl}) nach $I_{D}$ umgestellt, gilt:
\begin{equation} \label{eq:idgauß}  % Ist diese Gleichung überhaupt richtig? Überprüfen wenn Zeit
  \begin{split}
  I_{D} {} &= I_{\text{Zyl}} - m_{\text{Zyl}}  \left( \frac{3 r^2 + h^2}{6} \right) \\
    &\stackrel{\ref{var:aundb}}{=} \frac{b \cdot D}{4 \cdot \pi^2} - m_{\text{Zyl}}  \left( \frac{3 r^2 + h^2}{6} \right)
  \end{split}
\end{equation}

Mit (\ref{fehler:gauß}) ergibt sich  für $\increment I_{D}$:

\begin{equation*}
  \increment I_{D} = \sqrt{\left(\frac{b}{4 \pi^2} \cdot  \increment D\right)^2 + \left(\frac{D}{4 \pi^2} \cdot  \increment b\right)^2}
\end{equation*}

%%%%%%%%%%%%%%%%%%%%%%%%%%%%%%%%%%%%
%%%%%%%%%%%%%%%%%%%%%%%%%%%%%%%%%%%%
%%%%%%%%%%%%%%%%%%%%%%%%%%%%%%%%%%%%
%%%%%%%%%%%%%%%%%%%%%%%%%%%%%%%%%%%%

Damit lässt sich der Fehler für das Eigenträgheitsmoment angeben.
Die Zylinder haben die Maße $h = 0.023 \, \unit{\meter}, \: d = 0.45 \, \unit{\meter} \text{ und } m = 0.0624 \, \unit{\kilo\gram}$.
Es gilt also:
\begin{equation*}
  I_{D} =  (-0.00588 \pm 0.020731) \, \unit{\kilo\gram\meter\squared}
\end{equation*}

Es ist also so klein, dass das Eigenträgheitsmoment in der folgenden Auswertung vernachlässigt wird.

%%%%%%%%%%%%%%%%%%%%%%%%%%%%%%%%%%%%
%%%%%%%%%%%%%%%%%%%%%%%%%%%%%%%%%%%%
%%%%%%%%%%%%%%%%%%%%%%%%%%%%%%%%%%%%
%%%%%%%%%%%%%%%%%%%%%%%%%%%%%%%%%%%%

\subsection{Trägheitsmomente von Kugel und Zylinder}
\label{sec:KugelundZylinder}

\subsubsection*{Der Zylinder}
Der betrachtete Zylinder hat folgende Maße: $m_1 = 0.3678 \, \unit{\kilo\gram}, \: d_1 = 0.0983 \, \unit{\meter}$
Damit errechnet sich der Theoriewert des Trägheitsmomentes:

\begin{align*}
  I_{\text{Zyl, theo}} &= 4.4425 10^{-4} \, \unit{\kilo\gram\meter\squared}
\end{align*}

Die experimentellen Werte sind in Tabelle (\ref{tab:SchwingungsdauerZylinder}) zu finden.

\begin{table}
  \centering
  \caption{Messung der Schwingungsdauer des Zylinders}
  \label{tab:SchwingungsdauerZylinder}
  \begin{tabular}{c c}
    \toprule
     Messung &  $T / \unit\second$ \\
    \midrule
              1 &        0.800 \\
              2 &        0.750 \\
              3 &        0.760 \\
              4 &        0.720 \\
              5 &        0.750 \\
              6 &        0.740 \\
              7 &        0.740 \\
              8 &        0.790 \\
              9 &        0.750 \\
             10 &        0.780 \\
    \bottomrule
  \end{tabular}
\end{table}

Damit lässt sich die Schwingungsdauer angeben:
\begin{equation*}
  T_{Zyl} = (0.758 \pm 0.02357) \, \unit{\second}
\end{equation*}


Der Fehler berechnet sich nach (\ref{eq:idgauß}). Es folgt:

\begin{equation} \label{eq:fehler:traeg}
  \increment I = \sqrt{\left( \frac{2 \cdot D \cdot  T \cdot \increment T}{4 \pi} \right)^2 + \left( \frac{T^2 \cdot \increment D}{4 \pi}\right)^2}
\end{equation}


Damit kann das Trägheitsmoment $I_{\text{Zyl}}$ nun angegeben werden.


%%%%%%%%%%%%%%%%%%%%%%%%%%%%%%%%%%%%
%%%%%%%%%%%%%%%%%%%%%%%%%%%%%%%%%%%%
%%%%%%%%%%%%%%%%%%%%%%%%%%%%%%%%%%%% Muss noch errechnet werden
%%%%%%%%%%%%%%%%%%%%%%%%%%%%%%%%%%%%

\begin{equation*}
  I_{\text{Zyl}} = (3.056 \pm 1.475) \cdot 10^{-4} \, \unit{\kilo\gram\meter\squared}
\end{equation*}

%%%%%%%%%%%%%%%%%%%%%%%%%%%%%%%%%%%%
%%%%%%%%%%%%%%%%%%%%%%%%%%%%%%%%%%%%
%%%%%%%%%%%%%%%%%%%%%%%%%%%%%%%%%%%%
%%%%%%%%%%%%%%%%%%%%%%%%%%%%%%%%%%%%

\subsubsection*{Die Kugel}

Die betrachtete Kugel hat die folgenden Eigenschaften: $m_2 = 1.1716 \, \unit{\kilo\gram}$ und $d = 14.645 \, \unit{\centi\meter}$.
Die Formel für das Trägheitsmoment einer Kugel wird auch hier nur angegeben:
\begin{equation*}
  I_{\text{K}} = \frac{2}{5} m r^2
\end{equation*}
Mit den Werten bedeutet das:
\begin{equation*}
  I_{\text{K, theo}} = 2.5128 \cdot 10^{-3} \, \unit{\kilo\gram\meter\squared}
\end{equation*}

\begin{table}[H]
  \centering
  \caption{Messung der Schwingungsdauer der Kugel}
  \label{tab:SchwingungsdauerKugel}
  \begin{tabular}{c c}
    \toprule
    Messung &  $T / \unit\second$ \\
    \midrule
              1 &        1.770 \\
              2 &        1.900 \\
              3 &        1.850 \\
              4 &        1.800 \\
              5 &        1.800 \\
              6 &        1.800 \\
              7 &        1.900 \\
              8 &        1.850 \\
              9 &        1.880 \\
             10 &        1.870 \\
    \bottomrule
    \end{tabular}
\end{table}

Damit ergibt sich aus Mittelwert und Standartabweichung für $T_{Kug}$:
\begin{equation*}
  T_{Kug} = (1.842 \pm 0.044) \, \unit\second
\end{equation*}

Woraus das Trägheitsmoment $I_{\text{Kug}}$ folgt:

%%%%%%%%%%%%%%%%%%%%%%%%%%%%%%%%%%%%
%%%%%%%%%%%%%%%%%%%%%%%%%%%%%%%%%%%%
%%%%%%%%%%%%%%%%%%%%%%%%%%%%%%%%%%%%
%%%%%%%%%%%%%%%%%%%%%%%%%%%%%%%%%%%%


\begin{equation*}
  I_{\text{Kug}} = (1.8048 \pm 0.8413) \cdot 10^{-3} \, \unit{\kilo\gram\meter\squared}
\end{equation*}

%%%%%%%%%%%%%%%%%%%%%%%%%%%%%%%%%%%%
%%%%%%%%%%%%%%%%%%%%%%%%%%%%%%%%%%%%
%%%%%%%%%%%%%%%%%%%%%%%%%%%%%%%%%%%%
%%%%%%%%%%%%%%%%%%%%%%%%%%%%%%%%%%%%


\subsection{Trägheitsmoment der Puppe}

\subsubsection{Abmessungen}

\begin{table}
  \centering
  \caption{Abmessungen der Puppe}
  \label{tab:Abmessungen}
  \begin{tabular}{llll}
    \toprule
    Bein (14,33 cm) & - & Arm (13,02 cm) & - \\
    \midrule
               ober &      unter &           ober &      unter \\
               1.54 &       1.64 &            1.3 &       1.24 \\
               1.64 &       1.72 &           1.31 &       1.37 \\
               1.73 &       1.42 &            1.4 &       1.42 \\
               1.84 &       1.41 &           1.24 &        1.3 \\
                1.8 &        1.3 &           1.22 &        1.2 \\
    \bottomrule
    \end{tabular}\\

    \begin{tabular}{llr}
      \toprule
      Körper (9,81 cm) & - &  Kopf (4,71 cm) \\
      \midrule
                  ober &      unter &             - \\
                  3.96 &       3.33 &            2.83 \\
                  4.24 &       5.43 &            2.97 \\
                   4.3 &        3.5 &            2.73 \\
                  3.94 &        3.3 &            2.60 \\
                  3.25 &       3.17 &            2.25 \\
      \bottomrule
      \end{tabular}
\end{table}

Die Puppe hat die folgende Masse: $m_{\text{Puppe}} = 0.1079 \, \unit{\kilo\gram}$.
Die Maße der Gliedmaßen der Puppe sind in Tabelle (\ref{tab:Abmessungen}) aufgetragen.
Diese Werte werden gemittelt, um die Gliedmaßen durch einen Zylinder zu approximieren (siehe Tabelle (\ref{tab:MittelwertGlieder})).

\begin{table}[H]
  \centering
  \caption{Mittelwerte der Glieder}
  \label{tab:MittelwertGlieder}
  \begin{tabular}{rrrrrrrr}
    \toprule
       & $r_{\text{Bein}} / \unit\meter$ &     - &     $r_{\text{Arm}} / \unit\meter$ &     - &    $r_{\text{Körper}} /  \unit\meter$ &     - &    $r_{\text{Kopf}} /  \unit\meter$\\
    \midrule
     & ober & unter & ober & unter& ober & unter & \\
    Mittelwert: & 0.008550 & 0.007490 & 0.006470 & 0.006530 & 0.019690 & 0.018730 & 0.013380 \\
    Standartabw:: & 0.000544 & 0.000783 & 0.000316 & 0.000404 & 0.001866 & 0.004243 & 0.001225 \\
    \bottomrule
    \end{tabular}
\end{table}

Im Folgenden wird das Volumen der Puppe errechnet.
Dazu wird das Volumen der einzelnen Zylinder berechnet.
Das Zylindervolumen ist gegeben durch:
\begin{equation*}
  V_{\text{Zyl}} = \pi r^2 h
\end{equation*}
Ebenfalls wird das Kugelvolumen benötigt, um den Kopf zu approximieren:
\begin{equation*}
  V_{\text{Kug}} = \frac{4}{3} \pi r^3
\end{equation*}

Um den Fehler anzugeben, ist die Gauß'sche Fehlerfortpflanzung (\ref{fehler:gauß}) 
\begin{equation}
  \increment V_{\text{Ges}} = \sqrt{(\increment V_{\text{Körper}})^2 + 4 \cdot (\increment V_{\text{Arm}})^2 
    + 4 \cdot (\increment V_{\text{Bein}})^2 + (\increment V_{\text{Kopf}})^2}
\end{equation}

erfoderlich, wobei die einzelnen Volumina durch 
\begin{align*}
  \increment V_{\text{Zyl}} &= \sqrt{4 \pi^2 \cdot r^2 \cdot h^2 \cdot (\increment r)^2} & 
  \increment V_{\text{Kug}} &= \sqrt{16 \pi^2 \cdot r^4 \cdot (\increment r)^2}
\end{align*}

gegeben sind. Damit lassen sich die Volumina der einzelnen Körperteile angeben (Tabelle (\ref{tab:VolGlieder})):

\begin{table}[H]
  \centering
  \caption{Volumen der Körperteile}
  \label{tab:VolGlieder}
  \begin{tabular}{rrrrr}
    \toprule
    & Oberschenkel &     Unterschenkel &     Oberarm &     Unterarm \\
    \midrule
    $V / \unit{\milli\cubic\meter}$  & 16.455016 & 12.627858 & 8.561294 & 8.720818 \\
    $\increment V / \unit{\milli\cubic\meter}$  &   2.094151 &  2.638733 & 0.835208 & 1.080353  \\
    \midrule
    &    Oberkörper &     Bauch & Kopf & \\
    \midrule
    $V / \unit{\milli\cubic\meter}$  & 59.742077 & 54.058556 & 10.03360 & \\ 
    $\increment V / \unit{\milli\cubic\meter}$  &  11.320854 & 24.490641 & 2.75676 & \\
    \bottomrule
    \end{tabular}
\end{table}

Daraus resultiert das Gesamtvolumen, wobei das Volumen der Arme und Beine doppelt gezählt wird:

\begin{equation*}
  V_{\text{Ges}} = \sum_i V_{i} \pm  \increment V_{\text{Ges}}= (170.199 \pm 55.954) \cdot 10^{-6} \unit{\cubic\meter}
\end{equation*}

Nun werden die einzelnen Anteile am Gesamtvolumen bestimmt.
Damit wird schließlich die Masse der einzelnen Teile bestimmt.
Die Volumenanteile und deren Fehler berechnen sich über folgende Formeln:

\begin{align*}
  A_{\text{Körperteil}} & = \frac {V_{\text{Körperteil}}} {V_{\text{Ges}}} \\
  \increment A_{\text{Körperteil}} & = \sqrt{\left(\frac{1}{V_{\text{Ges}}}\right)^2 
  \cdot \left(\left( \increment V_{\text{Körperteil}} \right)^2 + \left( V_{\text{Körperteil}} 
  \cdot \increment V_{\text{Ges}} \right)^2\right)}
\end{align*}

Damit ergeben sich die Anteile in Tabelle (\ref{tab:AnteileKörper}):

\begin{table}
  \centering
  \caption{Anteile der Glieder}
  \label{tab:AnteileKörper}
  \begin{tabular}{rrrrrrrr}
    \toprule
      & $A_{\text{Oberschenkel}}$ &     $A_{\text{Unterschenkel}}$ &     $A_{\text{Oberarm}}$ &     $A_{\text{Unterarm}}$    &    $A_{\text{Oberkörper}}$ &     $A_{\text{Bauch}}$ & $A_{\text{Kopf}}$ \\
    \midrule
    & 0.0966 & 0.074 & 0.050 & 0.051 & 0.351 & 0.317 & 0.058 \\
    $\increment A_i$ & 0.034 & 0.029 & 0.017 & 0.018 & 0.133 & 0.178 & 0.025 \\
    \bottomrule
  \end{tabular}
\end{table}

Nun werden die Massen der Gliedmaßen berechnet:
\begin{equation*}
  m_{\text{Körperteil}} = A_{\text{Körperteil}} \cdot M_{\text{Ges}}
\end{equation*}

Damit ergeben sich die Massen in Tabelle {\ref{tab:Massenanteile}}

\begin{table}
  \centering
  \caption{Massenanteile}
  \label{tab:Massenanteile}
  \begin{tabular}{rrrrrrrrr}
    \toprule
    & $m_{\text{OS}} / \unit{\kilo\gram}$ &     $m_{\text{US}} / \unit{\kilo\gram}$ &     $m_{\text{OA}} / \unit{\kilo\gram}$ &     $m_{\text{UA}} / \unit{\kilo\gram}$    &    $m_{\text{OK}} / \unit{\kilo\gram}$ &     $m_{\text{Bauch}} / \unit{\kilo\gram}$ & $m_{\text{Kopf}} / \unit{\kilo\gram}$ \\
    \midrule
    & 0.01043 & 0.00800 & 0.00542 & 0.00552 & 0.03787 & 0.03427 & 0.00636 \\
    $\increment m_i / \unit{\kilo\gram}$ & 0.00367 & 0.00311 & 0.00186 & 0.00194 & 0.01437 & 0.01918 & 0.00272 \\ 
    \bottomrule
  \end{tabular}
\end{table}


Nun sind alle nötigen Parameter bestimmt, um die Trägheitsmomente der Puppe zu berechnen.

\subsubsection{Trägheitsmoment der Puppe: Pose 1}
Zunächst soll die \enquote{90° Pose} untersucht werden.
Diese ist ebenfalls in Abbildung (\ref{fig:pose1}) zu sehen.
In Tabelle (\ref{tab:PeriodendauerPose1}) sind die experimentellen Messwerte aufgetragen.


\begin{figure}[H]
  \centering
  \includegraphics[width=0.3\columnwidth]{pictures/puppe_rechter_winkel.jpg}
  \caption{90° Pose}
  \label{fig:pose1}
\end{figure}

\begin{table}[H]
  \centering
  \caption{Messwerte der Periodendauer: Pose 1}
  \label{tab:PeriodendauerPose1}
  \begin{tabular}{rrr}
    \toprule
     & $T / \unit\second$ &  $T / \unit\second$  \\
    \midrule
    $\varphi$ : & 90° & 120° \\
    \midrule
          & 0.90 &        0.86 \\
          & 0.87 &        0.90 \\
          & 0.86 &        0.85 \\
          & 0.85 &        0.90 \\
          & 0.88 &        0.80 \\
    \midrule
    $T_{\text{mittel}}$ : & 0.872 & 0.862 \\
    \bottomrule
    \end{tabular}
\end{table}

Daraus lässt sich die gemittelte Schwingungsdauer bestimmen:
\begin{equation*}
  T_E = (0.867 \pm 0.02934) \, \unit\second
\end{equation*}

Nun lässt sich das Trägheitsmoment berechnen.

\begin{equation*}
  \stackrel{(\ref{eq:Tumgestellt}),(\ref{fehler:gauß})}{\implies} I_E = 0.39985 \cdot 10^{-3}  % FEHLER FEHLT
  \, \unit{\kilo\gram\meter\squared} %Fehler angeben
\end{equation*}

Dieser Wert soll nun mit der theoretischen Erwartung verglichen werden.
Dafür wird zunächst das Trägheitsmoment der einzelnen Zylinder mithilfe der folgenden Gleichungen berechnet.
\begin{align*} \label{I_Puppe}
  I_{\text{Oberschenkel}} &= \frac{m_{\text{OS}} \cdot r_{\text{OS}}^2 + 2 \cdot m_{OS} \cdot r_{OS}^2} {2} \\
  I_{\text{Unterschenkel}} &= \frac{m_{\text{US}} \cdot r_{\text{US}}^2 + 2 \cdot m_{US} \cdot r_{US}^2} {2} \\
  I_{\text{Oberarm}} &= \frac{m_{\text{OA}} \cdot h_{\text{OA}^2} + 3 \cdot m_{\text{OA}} \cdot r_{\text{OK}}^2} {3} \\
  I_{\text{Unterarm}} &= \frac{m_{\text{UA}} \cdot h_{\text{UA}^2} + 3 \cdot m_{\text{UA}} \cdot \left( r_{\text{OK}} + h_{\text{OA}}\right)^2} {3} \\
\end{align*}

Insgesamt ergibt sich damit:
\begin{equation*}
  I_{\text{Ideal}} = 0.477969 \cdot 10^{-3} \, \unit{\kilo\gram\meter\squared}
\end{equation*}

\subsubsection{Trägheitsmoment der Puppe: Pose 2}

Nun wird das Trägheitsmoment der Figur in der \enquote{T-Pose} (siehe Abb. (\ref{fig:pose2})) untersucht.
Zunächst wird die theoretische Erwartung berechnet.
Aus dem Satz von Steiner (\ref{eq:SatzvSteiner}) folgen die Gleichungen:
\begin{align*}
  I_{\text{Oberschenkel}} &= \frac{m_{\text{OS}} \cdot h_{\text{OS}}^2} {3} \\
  I_{\text{Unterschenkel}} &= \frac{m_{\text{US}} \cdot h_{\text{US}}^2 + 3 \cdot m_{US} \cdot h_{OS}^2} {3} \\
  I_{\text{Oberarm}} &= \frac{m_{\text{OS}} \cdot r_{\text{OS}^2} + 2 \cdot m_{\text{OA}} \cdot (r_{\text{OA}} + r_{\text{OK}})^2} {2} \\
  I_{\text{Unterarm}} &= \frac{m_{\text{UA}} \cdot r_{\text{UA}^2} + 2 \cdot m_{\text{UA}} \cdot (r_{\text{UA}} + r_{\text{OK}})^2} {2} \\
\end{align*}

Damit lässt sich das Träheitsmoment berechnen:
\begin{equation*}
  I_{\text{Ideal}} = 0.13192 \cdot 10^{-3} \, \unit{\kilo\gram\meter\squared}
\end{equation*}

\begin{figure}[H]
  \centering
  \includegraphics[width=0.3\columnwidth]{pictures/puppe_tpose.jpg}
  \caption{T-Pose}
  \label{fig:pose2}
\end{figure}

\begin{table}
  \centering
  \caption{Messwerte der Periodendauer: Pose 2}
  \label{tab:PeriodendauerPose2}
  \begin{tabular}{rrr}
    \toprule
    & $T / \unit\second$ &  $T / \unit\second$  \\
    \midrule
    $\varphi$ : & 90° & 120° \\
    \midrule
          & 0.64 &        0.70 \\
          & 0.70 &        0.67 \\
          & 0.66 &        0.68 \\
          & 0.67 &        0.70 \\
          & 0.65 &        0.71 \\
    \midrule
    $T_{\text{mittel}}$ : & 0.664 & 0.692 \\
    \bottomrule
    \end{tabular}
\end{table}

Daraus lässt sich die gemittelte Schwingungsdauer bestimmen:
\begin{equation*}
  T_E = (0.678 \pm 0.02271) \, \unit\second
\end{equation*}

Daraus wiederum analog wie bei Pose 1 das Trägheitsmoment:

\begin{equation*}
  \stackrel{(\ref{eq:Tumgestellt}),(\ref{fehler:gauß})}{\implies} I_E = 0.24452  \cdot 10^{-3} % FEHLER FEHLT
  \, \unit{\kilo\gram\meter\squared} %Fehler angeben
\end{equation*}
