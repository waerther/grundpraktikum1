\section{Auswertung}
\label{sec:Auswertung}

\subsection{Vorbereitung}\label{subsec:Vorbereitung}
Um die Messungen auswerten zu können, muss zunächst das Eigenträgheitsmoment $I_{D}$ und die Winkelrichtgröße $D$ berechnet werden.

Mit (\ref{eq:drehmoment}) und (\ref{eq:drehmomentwinkel}) ergibt sich:
\begin{equation}
D = \frac{\lvert F \rvert \lvert r \rvert \sin{\vartheta}}{\varphi} \stackrel{F,r > 0}{=} \frac{F r \sin{\vartheta}}{\varphi} 
\stackrel{\vartheta = \frac{\pi}{2}}{=} \frac{F \cdot r}{\varphi}
\end{equation}

In Tabelle (\ref{tab:Winkelrichtgröße}) sind die entsprechenden Messwerte und die berechneten Winkelrichtgrößen.

\begin{table}
\centering
\caption{Messdaten zu Winkelrichtgröße, $r = 20 \unit{\centi\meter}$}
\label{tab:Winkelrichtgröße}
\begin{tabular}{c c c}
  \toprule
  $\varphi / °$  &  $F / \unit\newton$ & $D / \unit{\newton\meter}$ \\
  \midrule
              20 &        0.022 &     0.012605 \\
              30 &        0.054 &     0.020626 \\
              40 &        0.077 &     0.022059 \\
              50 &        0.092 &     0.021085 \\
              60 &        0.120 &     0.022918 \\
              70 &        0.144 &     0.023573 \\
              80 &        0.162 &     0.023205 \\
              90 &        0.188 &     0.023937 \\
             100 &        0.190 &     0.021772 \\
             110 &        0.200 &     0.020835 \\
             120 &        0.230 &     0.021963 \\
  \bottomrule
\end{tabular}
\end{table}

Es ergibt sich der Mittelwert:
\begin{align*}
  D_{mittel} = D  & = 0.021325364366313267 \pm 0.002952928161464375 \unit{\newton\meter}\\
  \implies D &  = 0.021 \pm 0.00295 \unit{\newton\meter}
\end{align*}

