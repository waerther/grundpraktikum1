\section{Theorie}
\label{sec:Theorie}

Als ein gekoppeltes Schwingsystem sind zwei durch eine Feder verbundene Fadenpendel schwer zu untersuchen sind.
Es wird daher auf elektrische Schwingkreise zurückgegriffen, da Frequenz und Amplitude dabei leichter zu bestimmen sind.

\subsection {Verhalten kapazitiv gekoppelter Schwingkreise}

Es werden zwei identische Schwingkreise betrachtet, die durch eine Kapazität $C_{K}$ miteinander verbunden sind.

\begin{figure} 
    \centering
    \includegraphics[width=10cm] {pictures/prinzipschaltbild.png}  
    \caption{Schaltung kapazitiv gekoppelter Schwingkreise. \cite{v355}}
    \label{fig:prinzipschaltbild}
\end{figure} 

Mithilfe der \textit{Kirchhoffschen Gesetze} ergibt sich im Knotenpunkt A
\begin{equation}
    I_{1} = I_{K} + I_{2}
\end{equation}

Für Masche 1 und 2 ergibt sich
\begin{equation}
    U_{{1,2}_{C}} + U_{{1,2}_{L}} + U_{K} = 0
\end{equation}

In dieses Gleichungssystem setzt man die Beziehungen
\begin{align} 
    U_{C} &= \frac{1}{C} \dif{t} \int I  & \mbox{\centering und} && U_{L} = L \dot{I} 
\end{align}

Damit erhält man für die beiden Maschen mit Differentiation nach $t$ die Differentialgleichungen
\begin{align}  
    L {\dot{I}}_{1} + \frac{1}{C} I_{1} + \frac{1}{C_{K}} (I_{1} - I_{2})  = 0 \label{eq:dgl_1} \\
    L {\dot{I}}_{2} + \frac{1}{C} I_{2} + \frac{1}{C_{K}} (I_{1} - I_{2})  = 0 \label{eq:dgl_2}
\end{align}

Addition und Subtraktion der Gleichungen (\ref{eq:dgl_1}) und (\ref{eq:dgl_2}) ergibt die entkoppelten 
Differentialgleichungen für die neuen Variablen ($I_{1}+I_{2}$) und ($I_{1}-I_{2}$)
\begin{align} 
    L \frac {\dif{^2}} {\dif{t^2}} (I_{1} + I_{2}) + \frac{1}{C} (I_{1} + I_{2})  = 0 \label{eq:dgl_1_entk} \\
    L \frac {\dif{^2}} {\dif{t^2}} (I_{1} - I_{2}) + \left( \frac{1}{C} + \frac{2}{C_{K}} \right) (I_{1} - I_{2})  = 0 \label{eq:dgl_2_entk}
\end{align}

Diese werden durch
\begin{align}
    (I_{1} + I_{2})(t) &= (I_{1,0} + I_{2,0}) \cos(I_{1} + I_{2}) \label{eq:dgl_1_sumsol} \\
    (I_{1} - I_{2})(t) &= (I_{1,0} - I_{2,0}) \cos \left( \frac {t} {{\sqrt [] {L \left( \frac{1}{C} + \frac{2}{C_{K}} \right)^{-1} }}} \right) \label{eq:dgl_2_sumsol}
\end{align}

gelöst. Durch erneute Addition und Subtraktion von (\ref{eq:dgl_1_sumsol}) und (\ref{eq:dgl_2_sumsol}) ergeben sich die Lösungen 
der ursprünglichen Ströme $I_{1}$ und $I_{2}$
\begin{align}
    I_{1}(t) &= \frac{1}{2} (I_{1,0} + I_{2,0}) \cos(2 \pi \nu^+ t)  +  \frac{1}{2} (I_{1,0} - I_{2,0}) \cos(2 \pi \nu^- t) \label{eq:dgl_1_sol} \\
    I_{2}(t) &= \frac{1}{2} (I_{1,0} + I_{2,0}) \cos(2 \pi \nu^+ t)  -  \frac{1}{2} (I_{1,0} - I_{2,0}) \cos(2 \pi \nu^- t) \label{eq:dgl_2_sol} 
\end{align}

wobei 
\begin{align}
    \nu^+ &= \frac {1} {2 \pi \sqrt[]{LC}} & \mbox{\centering und} && \nu^- = \frac {t} {2 \pi \sqrt [] {L \left( \frac{1}{C} + \frac{2}{C_{K}} \right)^{-1} }}
\end{align}

die Schwingunsfrequenzen von (\ref{eq:dgl_1_sumsol}) und (\ref{eq:dgl_2_sumsol}) darstellen. \\
\\
Sind zu Beginn die Amplituden beider Schwingkreise gleich groß ($I_{1,0} = I_{2,0}$), so entfallen bei (\ref{eq:dgl_1_sol}) und (\ref{eq:dgl_2_sol})
die jeweils letzten Terme. In diesem Fall schwingen die beiden Schwingkreise gleichphasig mit der Schwingunsfrequenz $\nu^+$, am
Koppelkondensator $C_{K}$ leigt dabei nie eine Spannung an. Bei gegengleichen Amplituden ($I_{1,0} = -I_{2,0}$) schwingen die Schwingkreise gegenphasig
mit der erhöhten Schwingunsfrequenz $\nu^-$ und die Spannung an $C_{K}$ wird maximal. Diese beiden Schwingungstypen werden \textit{Fundamentalschwingungen} genannt. \\
\\
Wird hingegen nur einer der beiden Schwingkreise ausgelenkt ($I_{1,0} \neq 0, I_{2,0} = 0$), so ergeben sich Schwebungen.
\begin{align}
    I_{1}(t) &= I_{1,0} \cos(\pi (\nu^+ + \nu^-) t) \cos(\pi (\nu^+ - \nu^-) t) \label{eq:dgl_1_sol_schweb} \\
    I_{2}(t) &= I_{1,0} \sin(\pi (\nu^+ + \nu^-) t) \sin(\pi (\nu^+ - \nu^-) t) \label{eq:dgl_2_sol_schweb} 
\end{align}

Der Verlauf der Ströme $I_{1}(t)$ und $I_{2}(t)$ werden in Abbildung (\ref{fig:schwebung}) dargestellt.

\begin{figure} 
    \centering
    \includegraphics[width=10cm] {pictures/schwebung.png} 
    \caption{Zeitlicher Verlauf der Ströme im Falle einer Schwebung. \cite{v355}}
    \label{fig:schwebung}
\end{figure} 

Die Amplitude der Schwingunsfrequenz ändert sich mit der \textit{Schwebungunsfrequenz} $\nu^- - \nu^+$, während das System mit der Frequenz
\begin{equation}
    \frac{1}{2} \left( \nu^+ + \nu^- \right) \approx \nu^+ 
\end{equation}
schwingt.


\subsection {Abhängigkeit des Stromes von der Frequenz}

Werden die Schwingkreise durch eine von außen angelegte Sinusspannung angeregt (siehe Abblilfung \ref{fig:sinusspannung}), \\
\begin{figure} 
    \centering
    \includegraphics[width=10cm] {pictures/sinusspannung.png} 
    \caption{Schwingkreise mit Sinusgenerator. \cite{v355}}
    \label{fig:sinusspannung}
\end{figure} 

so ergeben sich mithilfe der \textit{Kirchhoffschen Maschenregel}
\begin{align}
    U &= (z_{C} + z_{L} + z_{C_{K}} + z_{R}) I_{1} - z_{C_{K}} I_{2} \label{eq:sin_masche_1} \\
    0 &= (z_{C} + z_{L} + z_{C_{K}} + z_{R}) I_{2} - z_{C_{K}} I_{1} \label{eq:sin_masche_2} 
\end{align}

mit den Impedanzen
\begin{align} 
    z_{C} &= \frac{1}{i \omega C} & z_{L} = i \omega L  && z_{R} = R
\end{align}