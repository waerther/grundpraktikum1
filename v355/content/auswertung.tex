\section{Auswertung}
\label{sec:Auswertung}

\subsection{Abweichung der Justierung}

Die in Schaltung 1 (siehe Abbildung \ref{fig:foto}) verbauten Kenngrößen sind sind in 
Tabelle \ref{tab:komponenten_schaltung1} zu sehen.
\begin{table}
    \centering
    \caption{Werte der in Schwingkreis 1 verbauten Komponenten}
    \label{tab:komponenten_schaltung1}
    \begin{tabular}{c c}
        \toprule
        Kompentente &  Wert \\
        \midrule
        L               & $23.954 \, \unit{\milli\henry}$   \\
        C               & $0.7932 \, \unit{\nano\farad}$    \\
        $C_{\text{Sp}}$ & $ 0.028 \, \unit{\nano\farad}$    \\
        R               & $ 48 \, \unit{\ohm}$              \\
        \bottomrule
    \end{tabular}
\end{table}

Wobei $C_{\text{Sp}}$ die Kapazität der realen Spule entspricht, wie der Abblildung 
\ref{fig:spulenkapazität} entnommen werden kann.
\begin{figure} 
    \centering
    \includegraphics[width=10cm] {pictures/spulenkapazität.png}  
    \caption{Berücksichtigung der Spulenkapazität $\text{C}_{\text{Sp}}$. \cite{v355}}
    \label{fig:spulenkapazität}
\end{figure} 

Die bei der Justierung gemessene Resonanzfrequenz des ersten Schwingkreises beträgt
\begin{equation*}
    \text{C}_{\text{gemessen}} = \qty{35.7}{\kilo\hertz} \, .
\end{equation*}

Mithilfe von (\ref{eq:resonanzfrequenzen}) lässt sich die theoretische Fundamentalfrequenz 
von Schwingkreis 1 errechnen:
\begin{equation}
    \nu^+ = \frac{1}{2 \pi} \cdot \frac{1}{\sqrt{L C}}= \qty{36.51}{\kilo\hertz}
    \quad \text {bzw.} \quad
    \nu^+ = \frac{1}{2 \pi} \cdot \frac{1}{\sqrt{L\left(C+C_{\mathrm{SP}}\right)}}= \qty{35.88}{\kilo\hertz} 
\end{equation}

Die Ergebnisse zwischen Messung und Theoriewerte stimmen mit geringer Abweichung
überein (siehe Tabelle \ref{tab:abweichung_schaltung1}).
\begin{table}
    \centering
    \caption{Werte der in Schwingkreis 1 verbauten Komponenten}
    \label{tab:abweichung_schaltung1}
    \begin{tabular}{c c}
        \toprule
        {Betrachtete Kapazität $\mathbin{/} \, \unit{\nano\farad}$} &
        {Abweichung $\mathbin{/} \, \unit{\percent}$} \\
        \midrule
        C = 0.7932                                   & 2.3  \\
        $\text{C} + \text{C}_{\text{Sp}} = 0.8212$   & 0.5  \\
        \bottomrule
    \end{tabular}
\end{table}


\subsection{Verhältnis Schwingung- und Schwebungsfrequenz}

Die Verhältnisse der Schwingungs- und Schwebungsfrequenz lassen sich anhand der
Anzahl an Schwingungen pro Schwebung bestimmen (siehe Tabelle \ref{tab:maxima}).
\begin{table} [H]
    \centering
    \caption{Anzahl der bei der Messung ersichtlichen Maxima/Minima der Schwebungen im Vergleich mit dem Theoriewert.}
    \label{tab:maxima}
    \begin{tabular}{c c c c}
        \toprule
        ${C_\text{K}} \mathbin{/} \unit{\nano\farad}$ &  Maxima &
        $n$ & $\increment n_{\text{rel}} \mathbin{/} \unit{\percent}$ \\
        \midrule
            2.19 &       4  &  2.14 & 46.50 \\
            2.86 &       5  &  4.55 &  9.01 \\
            4.74 &       7  &  6.94 &  0.90 \\
            6.86 &       9  &  9.62 &  6.85 \\
            8.18 &      10  & 11.28 & 12.82 \\
            9.99 &      11  & 13.56 & 23.31 \\
            12.00 &      12 & 16.10 & 34.14 \\
        \bottomrule
    \end{tabular}
\end{table}

Die Frequenzen werden in Tabelle \ref{tab:schwebung} mithilfe von (\ref{eq:resonanzfrequenzen}), 
(\ref{eq:schwebung}) und (\ref{eq:schwingung}) theoretisch berechnet. Der Koppelkondensator
$C_\text{K}$ besitzt dabei einen Fehler von 0.3 $\unit{\percent}$, der sich wie in Kapitel 
\ref{sec:Fehlerrechnung} beschrieben fortpflanzt.
\begin{table}
    \centering
    \caption{   Berechnung der Schwingungs- und Schwebungsfrequenz anhand der Theoriewerte
                in Abhängigkeit der Kapazität $C_\text{K}$. }
    \label{tab:schwebung}
    \begin{tabular}{c c c c c}
        \toprule
        ${C_\text{K}} \mathbin{/} \unit{\nano\farad}$ &
        $\nu^+ \mathbin{/} \unit{\kilo\hertz}$ &
        $\nu^- \mathbin{/} \unit{\kilo\hertz}$ &
        $\nu_\text{Schwingung} \mathbin{/} \unit{\kilo\hertz}$ &
        $\nu_\text{Schwebung} \mathbin{/} \unit{\kilo\hertz}$ \\
        \midrule
         0.997${}\pm{}$0.003 &    36.51 & 58.77${}\pm{}$0.050 &    47.64${}\pm{}$0.027 &   22.26${}\pm{}$0.050 \\
          2.86${}\pm{}$0.009 &    36.51 & 45.53${}\pm{}$0.024 &    41.02${}\pm{}$0.012 &    9.02${}\pm{}$0.024 \\
          4.74${}\pm{}$0.014 &    36.51 & 42.18${}\pm{}$0.016 &    39.35${}\pm{}$0.008 &    5.67${}\pm{}$0.016 \\
          6.86${}\pm{}$0.021 &    36.51 & 40.51${}\pm{}$0.011 &    38.51${}\pm{}$0.006 &    4.00${}\pm{}$0.011 \\
          8.18${}\pm{}$0.025 &    36.51 & 39.90${}\pm{}$0.010 &    38.20${}\pm{}$0.005 &    3.39${}\pm{}$0.010 \\
          9.99${}\pm{}$0.030 &    36.51 & 39.30${}\pm{}$0.008 &    37.91${}\pm{}$0.004 &    2.79${}\pm{}$0.008 \\
         12.00${}\pm{}$0.040 &    36.51 & 38.85${}\pm{}$0.007 &    37.68${}\pm{}$0.003 &    2.34${}\pm{}$0.007 \\
        \bottomrule
    \end{tabular}
\end{table}

Die Größe $n$ in Tabelle \ref{tab:maxima} beschreibt das Verhältnis zwischen Schwingungs- und 
Schwebungsfrequenz. Somit gibt es die Anzahl der Maxima bzw. Minima innerhalb der Schwebung an. 
Die Maxima und Minima wechseln nach jeder $\pi$ Periode der Schwebung, sodass die Anzahl der Maxima
immer $\pi$ periodisch zwischen den zwei Werten schwankt. Die Berechnung von $n$ erfolgt dabei mit
\begin{equation}
    n=\frac{\nu^+ + \nu^-}{2\left(\nu^- - \nu^+\right)}=\frac{\nu_{\text{Schwingung}}}{\nu_{\text{Schwebung}}} \, .
\end{equation}

Zu beachten ist, dass die Messwerte
ganzzahlige Maxima darstellen, während die Theoriewerte das exakte Verhältnis von
Schwingungs- und Schwebungsfrequenz angeben. Die Berechnung der Abweichung $\increment n_{\text{rel}}$ erfolgt durch
\begin{equation}
    \increment n_{\text{rel}}=\frac{\increment n}{n}=\frac{\left|n-n_{t}\right|}{n} \, .
\end{equation}


\subsection{Die Frequenzen der Fundamentalschwingungen}

Die Ergebnisse der Messungen sind in Tabelle \ref{tab:vergleich} zu sehen, während die
theoretischen Frequenzen der Fundamentalschwingungen bereits bei der
Ermittelung der Verhältnisse zwischen Schwingungs- und Schwebungsfrequenz berechnet
wurden (siehe Tabelle \ref{tab:schwebung}). Der dabei auftretende Fehler wird in der Diskussion behandelt.
\begin{table}
    \centering
    \caption{   Vergleich der Theorie und Messwerte der Frequenzen der Fundamental-
                schwingungen. Die dabei genutzten Werte stammen aus Tabelle
                \ref{tab:maxima} und \ref{tab:schwebung}.}
    \label{tab:vergleich}
    \begin{tabular}{c c c c c}
    \toprule
    & \multicolumn{2}{c}{Theoriewerte} & \multicolumn{2}{c}{Messwerte} \\
    \cmidrule(lr){2-3}\cmidrule(lr){4-5}
    {${C_\text{K}} \mathbin{/} \unit{\nano\farad}$} &
    {$\nu^+ \mathbin{/} \unit{\kilo\hertz}$} & {$\nu^- \mathbin{/} \unit{\kilo\hertz}$} &
    {$\nu^+ \mathbin{/} \unit{\kilo\hertz}$} & {$\nu^- \mathbin{/} \unit{\kilo\hertz}$} \\
    \midrule
    0.997${}\pm{}$0.003 &    36.51 & 58.77${}\pm{}$0.050 &    35.1 &   55.8 \\
     2.86${}\pm{}$0.009 &    36.51 & 45.53${}\pm{}$0.024 &    35.1 &   46.1 \\
     4.74${}\pm{}$0.014 &    36.51 & 42.18${}\pm{}$0.016 &    35.1 &   40.7 \\
     6.86${}\pm{}$0.021 &    36.51 & 40.51${}\pm{}$0.011 &    35.1 &   39.2 \\
     8.18${}\pm{}$0.025 &    36.51 & 39.90${}\pm{}$0.010 &    35.1 &   38.9 \\
     9.99${}\pm{}$0.030 &    36.51 & 39.30${}\pm{}$0.008 &    35.1 &   38.0 \\
    12.00${}\pm{}$0.040 &    36.51 & 38.85${}\pm{}$0.007 &    35.1 &   37.5 \\
    \bottomrule
    \end{tabular}
\end{table}

Die Frequenz der Fundamentalschwingungen $\nu^-$ findet sich grafisch visualisiert in der
Abbildung \ref{fig:plot1}.
\begin{figure} [H]
    \centering
    \includegraphics[width=0.8\textwidth]{plot1.pdf}
    \caption{Theorie- und Messwerte der Fundamentalschwingung $\nu^-$ in Abhängigkeit
    der Kapazität ${C_\text{K}}$.}
    \label{fig:plot1}
\end{figure}


\subsection{Frequenzabhängige Stromstärke}
