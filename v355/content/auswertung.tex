\section{Auswertung}
\label{sec:Auswertung}

\subsection{Abweichung der Justierung}

Die in Schaltung 1 (siehe Abbildung \ref{fig:foto}) verbauten Kenngrößen sind sind in 
Tabelle \ref{tab:komponenten_schaltung1} zu sehen.
\begin{table}
    \centering
    \caption{Werte der in Schwingkreis 1 verbauten Komponenten}
    \label{tab:komponenten_schaltung1}
    \begin{tabular}{c c}
        \toprule
        Kompentente &  Wert \\
        \midrule
        L               & $23.954 \, \unit{\milli\henry}$   \\
        C               & $0.7932 \, \unit{\nano\farad}$    \\
        $C_{\text{Sp}}$ & $ 0.028 \, \unit{\nano\farad}$    \\
        R               & $ 48 \, \unit{\ohm}$              \\
        \bottomrule
    \end{tabular}
\end{table}

Wobei $C_{\text{Sp}}$ die Kapazität der realen Spule entspricht, wie der Abblildung 
\ref{fig:spulenkapazität} entnommen werden kann.
\begin{figure} 
    \centering
    \includegraphics[width=10cm] {pictures/spulenkapazität.png}  
    \caption{Berücksichtigung der Spulenkapazität $\text{C}_{\text{Sp}}$. \cite{v355}}
    \label{fig:spulenkapazität}
\end{figure} 


Die bei der Justierung gemessene Resonanzfrequenz des ersten Schwingkreises beträgt
\begin{equation*}
    \text{C}_{\text{gemessen}} = \qty{35.7}{\kilo\hertz} \, .
\end{equation*}

Mithilfe von (\ref{eq:resonanzfrequenzen}) lässt sich die theoretische Fundamentalfrequenz 
von Schwingkreis 1 errechnen:
\begin{equation}
    \nu_{+}=\frac{1}{2 \pi} \cdot \frac{1}{\sqrt{L C}}= \qty{36.51}{\kilo\hertz}
    \quad \text {bzw.} \quad
    \nu_{+}=\frac{1}{2 \pi} \cdot \frac{1}{\sqrt{L\left(C+C_{\mathrm{SP}}\right)}}= \qty{35.88}{\kilo\hertz} 
\end{equation}

Die Ergebnisse zwischen Messung und Theoriewerte stimmen mit geringer Abweichung
überein (siehe Tabelle \ref{tab:abweichung_schaltung1}).
\begin{table}
    \centering
    \caption{Werte der in Schwingkreis 1 verbauten Komponenten}
    \label{tab:abweichung_schaltung1}
    \begin{tabular}{c c}
        \toprule
        {Betrachtete Kapazität $\mathbin{/} \, \unit{\nano\farad}$} &
        {Abweichung $\mathbin{/} \, \unit{\percent}$} \\
        \midrule
        C = 0.7932                                   & 2.3  \\
        $\text{C} + \text{C}_{\text{Sp}} = 0.8212$   & 0.5  \\
        \bottomrule
    \end{tabular}
\end{table}
