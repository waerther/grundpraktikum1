\section{Diskussion}
\label{sec:Diskussion}

Bereits bei der Justierung der Messgeräte fallen die genauen Messwerte für die erste
Eigenfrequenz auf. Denn die Abweichung beträgt auch bei Vernachlässigung der Kapazität
der Spule nur  $\qty{2.3}{\percent}$, wie es schon in Tabelle \ref{tab:abweichung_schaltung1}
dargestellt wird. Der Grund für diese geringe Abweichung besteht darin, dass nur in einem
schmalen Frequenzbereich die gewünschte Lissajous-Figur auf dem Oszilloskop zu sehen ist. 
Denn sobald sich die Frequenz ein wenig von der Resonanzfrequenz unterscheidet, ist die geradenförmige
Lissajous-Figur, welche eine Phasenverschiebung von 0 bzw. $\pi$ indiziert, nicht mehr
auf dem Oszilloskop nachweisbar. Somit erfordert diese Einstellung zwar ein gewisses
Feingefühl bei der Einstellung der Frequenz, allerdings resultiert dies auch in genauen
Werten. Die Schaltungsbauteile scheinen aufgrund der gemessen Größen die Wirklichkeit
der angegebenen Größen dieser zu bestätigen.\\
\\
Bei der Betrachtung der Verhältnisse der Schwingungsfrequenzen pro Schwebung wird ersichtlich,
dass sie zwar von den Theoriewerten abweichen, jeoch sie sich noch in einem absehbaren Rahmen halten
(siehe Tabelle \ref{tab:maxima}), wenn die Tatsache berücksichtigt wird, dass einzelne Schwingungen
an den Rändern einer Schwebung häufig nicht eindeutig zugeordnet werden können.
Denn es werden bei der Schwebung nicht zwei ganze Schwebungen als eine Wellenlänge
angesehen, sondern nur die Länge einer einzigen „Perle“. Daraus resultiert die Tatsache,
dass bei der Schwebung, effektiv gesehen, die doppelte Frequenz zum Vergleich mit der
Schwingungsfrequenz verwendet wird.\\
\\
Im Gegensatz zu den oben diskutierten Messresultaten kommt es bei der Messung der
Resonanzfrequenzen zu ähnlichen Abweichungen, wie der Vergleich
mit Tabelle \ref{tab:vergleich} zeigt oder in Abbildung \ref{fig:plot1} zu erkennen ist. 
Dabei weichen die Messwerte alle um einen ähnlichen Wert nach unten relativ zu den Theoriewerten ab. 
Der Schluss, der daraus zu ziehen ist, dass ein systematischer Fehler auftritt. 
Doch auch der systematische Fehler kann nicht sehr groß sein, da die maximale Abweichung in
Tabelle \ref{tab:vergleich} max{ $\{ \increment \nu^-_{\text{rel}} \} = \qty{5.33}{\percent}$ beträgt.