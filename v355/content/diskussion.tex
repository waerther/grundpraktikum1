\section{Diskussion}
\label{sec:Diskussion}

Bereits bei der Justierung der Messgeräte fallen die genauen Messwerte für die erste
Eigenfrequenz auf. Denn die Abweichung beträgt auch bei Vernachlässigung der Kapazität
der Spule nur  $\qty{2.3}{\percent}$, wie es schon in Tabelle \ref{tab:abweichung_schaltung1}
dargestellt wird. Der Grund für diese geringe Abweichung besteht darin, dass nur in einem
schmalen Frequenzbereich die gewünschte Lissajous-Figur auf dem Oszilloskop zu sehen ist. 
Denn sobald sich die Frequenz ein wenig von der Resonanzfrequenz unterscheidet, ist die geradenförmige
Lissajous-Figur, welche eine Phasenverschiebung von 0 bzw. $\pi$ indiziert, nicht mehr
auf dem Oszilloskop nachweisbar. Somit erfordert diese Einstellung zwar ein gewisses
Feingefühl bei der Einstellung der Frequenz, allerdings resultiert dies auch in genauen
Werten. Die Schaltungsbauteile scheinen aufgrund der gemessen Größen die Wirklichkeit
der angegebenen Größen dieser zu bestätigen.\\
\\
Wie auch schon in der Auswertung angegeben wurde, stimmen ebenfalls die Verhält-
nisse der Schwingungsfrequenzen pro Schwebung gut mit den Theoriewerten überein.
Denn es wird bei der Schwebung nicht zwei ganze Schwebungen als eine Wellenlänge
angesehen, sondern nur die Länge einer einzigen „Perle“. Daraus resultiert die Tatsache,
dass bei der Schwebung, effektiv gesehen, die doppelte Frequenz zum Vergleich mit
Schwingungsfrequenz verwendet wird. Damit dann die Theoriewerte mit den Messwerten
verglichen werden können, werden im Folgenden die Anzahl der Maxima aus Tabelle 3
verdoppelt und den berechneten Verhältnissen aus Tabelle 4 gegenübergestellt.