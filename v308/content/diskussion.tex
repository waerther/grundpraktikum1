\section{Diskussion}
\label{sec:Diskussion}

Die Messwerte des Versuchs sind größtenteils von der Theorie hervorgesagt, jedoch gab es durchaus Abweichungen.
Im folgenden werden die Ursachen und die Abweichungen diskutiert.
Bei der Messung der langen Fehler sieht man, dass sich eine Art Supremum andeutet.
Dies ist auch zu erwarten, da das Magnetfeld innerhalb theoretisch homogen sein sollte.
Aufgrund der Messung erkennt man nur eine Tendenz, denn es wurden nicht ausreichend Messwerte innerhalb der Spule gemessen.
Das theoretische Supremum sollte laut Theorie bei $B_{\text{LS}} = 2.31 T$ liegen.
Das bedeutet, es ergibt sich eine Differenz mit dem zu letzt gemessenen Wert von ungefähr $5.63\%$.
