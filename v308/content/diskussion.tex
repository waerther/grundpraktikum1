\section{Diskussion}
\label{sec:Diskussion}

Die Messwerte des Versuchs sind größtenteils von der Theorie hervorgesagt, jedoch gab es durchaus Abweichungen.
Im folgenden werden die Ursachen und die Abweichungen diskutiert.
Bei der Messung der langen Spule sieht man, dass sich eine Art Supremum andeutet.
Dies ist auch zu erwarten, da das Magnetfeld innerhalb theoretisch homogen sein sollte.
Aufgrund der Messung erkennt man nur eine Tendenz, denn es wurden nicht ausreichend Messwerte innerhalb der Spule gemessen.
Das theoretische Supremum sollte bei $B_{\text{LS}} = 2.31 T$ liegen.
Das bedeutet, es ergibt sich eine Differenz mit dem zu letzt gemessenen Wert von ungefähr $5.63\%$. \\

Bei dem Helmholtzspulenpaar kann man die durch die Theorie hervorgesagte Homogenität im Bereich der Mitte gut beobachten.
Schaut man sich die Abbildungen an, ist eine klare Tendenz zur Homogenität zu erkennen.
Vergrößert man nun den Abstand, geht die "Ausdehnung" der Homogenität verloren.
Vergleicht man die Parabeln von $d_{1}$ und $d_{3}$, erkennt man deutlich die Reduzierung dieser Breite.
Beispielsweise gleicht $d_{3}$ fast schon einer Parabel.
Das liegt daran, dass das Verhältnis von Radien der Spulen zu Abstand immer suboptimaler wird.  \\

Bei der Ringspule wurden die Erwartungen beobachtet.
Es zeichnet sich eine Neukurve ab und es lässt sich eine deutliche Hysteresekurve beobachten.
Allerdings sind die Ergebnisse für die Koerzitivkraft und die Remanenz doch deutlich unterschiedlich nach der Umpolung.
Erwarten würde man, dass $B_{r,1} \approx B_{r,2}$ und $H_{c,1} \approx H_{c,2}$
Jedoch lässt sich dies nicht beobachten. 
Man beobachtet eine Abweichung von $\increment B = 56.16 \%$ und  $\increment H = 40 \%$