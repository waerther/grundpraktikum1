\section{Durchführung}
\label{sec:Durchführung}

\subsection{Lange Spule}
\label{sec:LangeSpule}

Zunächst wird das Magnetfeld $B$ einer langen Spule mit Windungszahl $N = 300$ und der Länge $L = 16.3$ gemessen.
Dafür wird es zunächst an ein Netzteil angeschlossen. Bei der Durchführung wurde dabei ein Strom $I = 1A$ angelegt.
Um das Magnetfeld zu messen wird der Halleffekt verwendet.
Dafür wird eine longitudinale Hall-Sonde benutzt.
Wir beginnen die Messung etwas außerhalb bei -0.02$m$ und mit einer Schrittgröße von 0.005$m$ wird die Sonde weiter reingeschoben.

\subsection{Helmholtzspulenpaar}
\label{sec:Spulenpaar}

Das Helmholtzspulenpaar besteht aus jeweils zwei Spulen, wobei jede Spule identisch ist.
Der Radius $R$ beträgt dabei 62.5$mm$ und die Windungszahl $N$ beträgt 100.
Die Spulen werden über ein Netzgerät mit dem Strom $I = 2.1A$ betrieben. 
Hier wird das Magnetfeld um das Zentrum der beiden Spulen und etwas außerhalb in Abhängigkeit des Abstandes gemessen.
Dabei wird eine transversale Hall-Sonde benutzt.
Die Messung wird mit verschiedenen Abständen durchgeführt.
Diese sind $d_{1} = 0.012m$, $d_{1} = 0.014m$ und $d_{1} = 0.016m$.

\subsection{Hysteresekurve einer Ringspule}
\label{sec:Hysteresekurve}

Hier soll die Hysteresekurve einer Ringspule gemessen werden.
Dafür wird eine Ringspule mit Windungszahl $N = 595$ und einem Luftspalt mit der Breite $d = 0.003m$ and das Netzteil angeschlossen.
Dabei wird der Strom in $1A$-Schritten auf $10A$ erhöht, beginnend von $I = 0$.
Danach wird er in der selben Schrittweite auf 0 reduziert.
Dann wird umgepolt und der Prozess wird erneut durchlaufen.
Am Ende wird nun erneut umgepolt und wieder auf $10A$ erhöht, um die Hysteresekurve einmal komplett zu durchlaufen.
Dabei wird erneut mit einer transversalen Hall-Sonde gemessen.
Diese wird dabei innerhalb des Luftspaltes angebracht.