\section{Durchführung}
\label{sec:Durchführung}

\subsection{Lange Spule}
\label{sec:LangeSpule}

Zunächst wird das Magnetfeld $B$ einer langen Spule mit Windungszahl $N = $ %%%%%%% und der Länge $L = $  %%%%  gemessen.
Dafür wird es zunächst an ein Netzteil angeschlossen. Bei der Durchführung wurde dabei ein Strom $I$ von  %%%% angelegt.
Um das Magnetfeld zu messen wird der Halleffekt verwendet. % Hier Verweis auf Theorie geben
Dafür wird eine longitudinale Hall-Sonde benutzt.
Wir beginnen die Messung etwas außerhalb bei %%% und mit einer Schrittgröße von %%% wird die Sonde weiter reingeschoben.

\subsection{Helmholtzspulenpaar}
\label{sec:Spulenpaar}

Das Helmholtzspulenpaar besteht aus jeweils zwei Spulen, wobei jede Spule identisch ist.
Der Radius $R$ beträgt dabei %%% und die Windungszahl $N$ beträgt %%%
Die Spulen werden über ein Netzgerät mit dem Strom $I = $ betrieben. 
Hier wird das Magnetfeld um das Zentrum der beiden Spulen und etwas außerhalb in Abhängigkeit des Abstandes gemessen.
Dabei wird eine transversale Hall-Sonde benutzt.
Die Messung wird mit verschiedenen Abständen durchgeführt.
Diese sind %%%% $d_{1} = $, $d_{1} = $ und $d_{1} = $.

\subsection{Hysteresekurve einer Ringspule}
\label{sec:Hysteresekurve}

Hier soll die Hysteresekurve einer Ringspule gemessen werden.
Dafür wird eine Ringspule mit Windungszahl $N = $ und einem Luftspalt mit der Breite $d = $ and das Netzteil angeschlossen.
Dabei wird der Strom in $1A$-Schritten auf $10A$ erhöht, beginnend von $I = 0$.
Danach wird er in der selben Schrittweite auf 0 reduziert.
Dann wird umgepolt und der Prozess wird erneut durchlaufen.
Am Ende wird nun erneut umgepolt und wieder auf $10A$ erhöht, um die Hysteresekurve einmal komplett zu durchlaufen.
Dabei wird erneut mit einer transversalen Hall-Sonde gemessen.
Diese wird dabei innerhalb des Luftspaltes angebracht.