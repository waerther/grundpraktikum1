\section{Diskussion}
\label{sec:Diskussion}

Auffällig ist, dass die Werte der Zeitkonstanten nah bei einander liegen und generell die
experimentellen Graphen meist den theoretischen, erwarteten Graphen sehr ähnlich sind.
Die Entladekurve des Kondensators entspricht einem erwarteten exponentiellen Abfall der
Spannung. Die Amplitude der Kondensatorspannung sinkt mit zunehmender Frequenz
ebenfalls wie erwartet exponentiell. Auch die Phasenverschiebung bleibt wie erwartet
zwischen 0 und $\pi/2$. Die Zeitkonstante kann über die drei Messreihen ausgerechnet
werden. Der genauen Wert für der Zeitkonstante kann nicht weiter berechnet werden, da
der Ohmsche Widerstand R des Widerstand und die Kapazität C des Kondensators nicht
gegeben sind und keine Messgeräte vorhanden sind. Die gemessene Zeitkonstante über
die Methode der Entladung des Kondensators ist 2.03 mal kleiner als die Zeitkonstante
aus der Messreihe zur Methode der Amplitude in Abhängigkeit von der Freqenz. Die
Zeitkonstante aus der Messung der Amplitude in Abhängigkeit von der Freqenz ist
2.69 mal kleiner als der Wert aus der Methode über die Phasenverschiebung zwischen
Generator- und Kondensatorspan- nung. Diese Abweichungen können unter anderem
daher kommen, dass eine geringer Ablesefehler auf dem Oszilloskop nicht ausgeschlossen
werden kann.