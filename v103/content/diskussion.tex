\section{Diskussion}
\label{sec:Diskussion}
Die Werte bei der einseitig eingespannten Messung sind durchaus vielversprechend.
Die Ergebnisse werden nun mit dem Literaturwert von Kupfer verglichen, welcher laut der Quelle {\cite{demtroeder1}} bei
\begin{equation*}
  E_\text{Lit} = 125 \cdot 10^9 \frac{\unit\newton}{\unit\meter^2}
\end{equation*}
liegt. Bei dem runden Stab ergibt sich eine Abweichung von
\begin{equation*}
  \increment E = \left|\frac{108.4 - 125}{125}\right| \cdot 100 = 13.28 \%
\end{equation*}
zum Theoriewert.
Die Abweichung bei dem eckigen berechnet sich zu
\begin{equation*}
    \increment E = \left|\frac{124.3 - 125}{125}\right| \cdot 100 = 0.56 \% \text{ .}
\end{equation*}
Bei Abweichungen von 13.28 \% und 0.56 \% kann von einer erfolgreichen Messung gesprochen werden.
Bei der beidseitigen Messung jedoch ergeben sich große Abweichungen.
Bei dem runden Stab ergibt sich eine Abweichung von 
\begin{equation*}
    \increment E = \left|\frac{158 - 125}{125}\right| \cdot 100 = 26.4 \% \text{ ,}
\end{equation*}
obwohl der selbe Stab verwendet wurde.
Bei dem eckigen Stab berechnet sich die Abweichung zu
\begin{equation*}
    \increment E = \left|\frac{81 - 125}{125}\right| \cdot 100 = 35.2 \% \text{ .}
  \end{equation*}
Die Messung ist weniger erfolgreich und die Standartabweichung ist mit 10 auch noch relativ hoch.
In Abbildung \ref{fig:plot4} wird ersichtlich, dass die beiden Messuhren wohl unterschiedlich gemessen haben,
trotz Nullmessung, denn die Diskrepanz zwischen den Werten ist groß.
Natürlich hat der Stab an einigen Stellen eine andere Beschaffenheit, jedoch sollte sich
dies eigentlich durch die Nullmessung der Uhren rauskürzen.
Dennoch ist die Messung grundsätzlich ein Erfolg gewesen, jedoch scheint es ein systematischen
Fehler bei der Messung bei beidsetiger Einspannung gegeben zu haben.
Das geringe Gewicht, welches bei der Belastung verwendet wurde, wurde zu gering gewählt.
Dadurch konnte der Stab bei beidseitiger Auflage nicht weit genug durchgebogen werden.
Es kann also nicht mit Sicherheit gesagt werden, dass es sich wirklich um das 
angenommene Material (Kupfer) handelt.