\section{Diskussion}
\label{sec:Diskussion}
Die Werte bei der einseitig eingespannten Messung sind durchaus vielversprechend
und die Abweichung zu dem angenommenen Material ist absolut im Rahmen.
Bei Abweichungen von 13.28 \% und 0.56 \% kann man von einer erfolgreichen Messung sprechen.
Bei der beidseitigen Messung jedoch ergaben sich große Abweichungen.
Bei dem runden Stab ergab sich eine Abweichung von 44 \%, obwohl der selbe Stab verwendet wurde.
Bei dem eckigen Stab ist das Ergebnis eigenltich ziemlich gut, trotzdem ist die Standartabweichung mit 11
ziemlich hoch, dennoch ist es im Rahmen.
Man sieht hier jedoch stark, dass die beiden Messuhren wohl unterschiedlich gemessen haben,
trotz Nullmessung, denn die Diskrepanz zwischen den Werten in Abbildung \ref{fig:plot4} ist doch ziemlich groß.
Natürlich hat der Stab an einigen Stellen eine andere Beschaffenheit, jedoch sollte sich
dies eigentlich durch die Nullmessung der Uhren rauskürzen.
Dennoch ist die Messung grundsätzlich ein Erfolg gewesen, jedoch scheint es ein systematischen
Fehler bei der Messung bei beidsetiger Einspannung gegeben zu haben.
Man kann trotzdem nicht mit Sicherheit sagen, dass es sich wirklich um das 
angenommene Material handelt.