\section{Auswertung}
\label{sec:Auswertung}

\subsection{Runder Stab, einseitige Einspannung}
Zunächst soll die Gesamtauslenkung bestimmt werden.
Dafür wird die Nullmessung von der Messung mit Gewicht abgezogen.
Außerdem werden die Stellen, an denen der Stab gemessen wird, in die Formel 
\begin{equation}
  f(x) = Lx^2 - \frac{x^3}{3}
\end{equation}
eingesetzt und dann als x Koordinate für den Plot verwendet.
Die Ergebnisse sind in Tabelle \ref{tab:runderstabeinseitig} aufgetragen.

Aus der Ausgleichrechnung für den Plot in Abbildung \ref{fig:plot1} mit dem Ansatz
\begin{equation*}
  y(x) = a + b \cdot x
\end{equation*}
gehen die Parameter
\begin{align*}
  a &= 0.0380 ± 0.0004 \frac{1}{\unit{\meter\squared}} & \text{und}& & b&= (0.1248 ± 0.0193) \cdot 10^{-3} \unit\meter\\
\end{align*}
hervor. Nun wird diese Gerade mit der Formel für den Elastizitätsmodul $E$ in (\ref{eq:E1}) verglichen, wobei für $I$ nach (\ref{eq:FlTreagKreis})
\begin{equation*}
  I = \frac{R^4 \pi}{2} = \frac{0.005^4 \pi}{2}
\end{equation*}
und es gilt
\begin{equation*}
  E = \frac{m \cdot g} {2 \cdot I \cdot a} = (108.4 \pm 1.1) 10^9 \frac{\unit\newton}{\unit\meter^2} \text{ .}
\end{equation*}

%%%%%%%%%%%%%%%%%%%%%%%%%%%%%%%%%%%%%%%%%%%%%%%%%%%
%%%%%%%%%%%%%%%%%%%%%%%%%%%%%%%%%%%%%%%%%%%%%%%%%%%
%%%%%%%%%%%%%%%%%%%%%%%%%%%%%%%%%%%%%%%%%%%%%%%%%%%

%Dieser Wert wird mit dem Literaturwert {\cite{demtroeder1}}
%\begin{equation*}
%  E_\text{Lit} = 125 \cdot 10^9 \frac{\unit\newton}{\unit\meter^2}
%\end{equation*}
%von Kupfer verglichen. Dabei ergibt sich eine Abweichung von
%\begin{equation*}
%  \increment E = \left|\frac{108.4 - 125}{125}\right| \cdot 100 = 13.28 \%
%\end{equation*}
%vom Theoriewert.

%%%%%%%%%%%%%%%%%%%%%%%%%%%%%%%%%%%%%%%%%%%%%%%%%%%
%%%%%%%%%%%%%%%%%%%%%%%%%%%%%%%%%%%%%%%%%%%%%%%%%%%
%%%%%%%%%%%%%%%%%%%%%%%%%%%%%%%%%%%%%%%%%%%%%%%%%%%

\begin{table}[H]
  \centering
  \caption{Messwerte der Auslenkung des runden Stabes bei einseitiger Einspannung.}
  \label{tab:runderstabeinseitig}
  \begin{tabular}{c c c c c}
    \toprule
      $x / 10 ^{-3} \unit\meter$ &  $D_0 (x) / 10^{-3} \unit\meter$ &
        $D_m (x) / 10^{-3} \unit\meter$ & $D(x) / 10^{-3} \unit\meter$ & $(Lx^2 - \frac{x^3}{3}) / 10^{-3} \unit\meter$\\
    \midrule
       30.000 & 0.000 & 0.040 & 0.040 &   0.522 \\
       50.000 & 0.020 & 0.200 & 0.180 &   1.433 \\
       70.000 & 0.006 & 0.280 & 0.274 &   2.777 \\
       90.000 & 0.120 & 0.350 & 0.230 &   4.536 \\
      110.000 & 0.200 & 0.530 & 0.330 &   6.695 \\
      130.000 & 0.290 & 0.715 & 0.425 &   9.239 \\
      150.000 & 0.380 & 0.980 & 0.600 &  12.150 \\
      170.000 & 0.495 & 1.200 & 0.705 &  15.413 \\
      190.000 & 0.640 & 1.460 & 0.820 &  19.013 \\
      210.000 & 0.700 & 1.710 & 1.010 &  22.932 \\
      230.000 & 0.830 & 2.010 & 1.180 &  27.155 \\
      250.000 & 0.920 & 2.310 & 1.390 &  31.667 \\
      270.000 & 1.070 & 2.610 & 1.540 &  36.450 \\
      290.000 & 1.160 & 2.950 & 1.790 &  41.489 \\
      310.000 & 1.350 & 3.300 & 1.950 &  46.769 \\
      330.000 & 1.440 & 3.630 & 2.190 &  52.272 \\
      350.000 & 1.610 & 3.980 & 2.370 &  57.983 \\
      370.000 & 1.765 & 4.350 & 2.585 &  63.887 \\
      390.000 & 1.925 & 4.710 & 2.785 &  69.966 \\
      410.000 & 2.090 & 5.120 & 3.030 &  76.205 \\
      430.000 & 2.280 & 5.510 & 3.230 &  82.589 \\
      450.000 & 2.430 & 5.910 & 3.480 &  89.100 \\
      470.000 & 2.540 & 6.340 & 3.800 &  95.723 \\
      490.000 & 2.830 & 6.690 & 3.860 & 102.443 \\
    \bottomrule
    \end{tabular}
\end{table}

\begin{figure}[H]
  \centering
  \includegraphics{pictures/Lineare Regression1.pdf}
  \caption{Plot des runden, einseitig eingespannten Stabes gegen die lin. Regression.}
  \label{fig:plot1}
\end{figure}

\subsection{Eckiger Stab, einseitige Einspannung}

Für den eckigen Stab wird analog vorgangen.
Die Messwerte für diese Messung sind in Tabelle \ref{tab:eckigerstabeinseitig} zu finden.
Die Ausgleichsgerade $y(x) = a \cdot x + b$ in Abbildung \ref{fig:plot2} hat die Parameter
\begin{align*}
  a &= 0.0254 ± 0.0003 \frac{1}{\unit{\meter\squared}} & \text{und}& & b&= (0.0401 ± 0.0138) \cdot 10^{-3} \unit\meter \text{ .} \\
\end{align*}
Das Flächenträgheitsmoment wird nach Formel (\ref{eq:FlTreagKreis}) bestimmt und liefert mit Höhe $H = 0.01 \unit\meter$
\begin{equation*}
  I_\text{Quadrat} = \frac{h^4} {12} = \frac{0.01^4} {12} \text{ .}
\end{equation*}
Nun kann das Elastizitätsmodul berechnet werden zu
\begin{equation*}
  E = (124.3 \pm 1.5) \cdot 10^{9} \text{ .}
\end{equation*}

%%%%%%%%%%%%%%%%%%%%%%%%%%%%%%%%%%%%%%%%%%%%%%%%%%%
%%%%%%%%%%%%%%%%%%%%%%%%%%%%%%%%%%%%%%%%%%%%%%%%%%%
%%%%%%%%%%%%%%%%%%%%%%%%%%%%%%%%%%%%%%%%%%%%%%%%%%%

%Vergleicht man das mit dem Literaturwert ergibt sich eine Abweichung von
%\begin{equation*}
%  \increment E = \left|\frac{124.3 - 125}{125}\right| \cdot 100 = 0.56 \% \text{ .}
%\end{equation*}

%%%%%%%%%%%%%%%%%%%%%%%%%%%%%%%%%%%%%%%%%%%%%%%%%%%
%%%%%%%%%%%%%%%%%%%%%%%%%%%%%%%%%%%%%%%%%%%%%%%%%%%
%%%%%%%%%%%%%%%%%%%%%%%%%%%%%%%%%%%%%%%%%%%%%%%%%%%

\begin{table}[H]
  \centering
  \caption{Messwerte der Auslenkung des eckigen Stabes bei einseitiger Einspannung.}
  \label{tab:eckigerstabeinseitig}
  \begin{tabular}{c c c c c}
    \toprule
    $x / 10 ^{-3} \unit\meter$ &  $D_0 (x) / 10^{-3} \unit\meter$ &
    $D_m (x) / 10^{-3} \unit\meter$ & $D(x) / 10^{-3} \unit\meter$ & $(Lx^2 - \frac{x^3}{3}) / 10^{-3} \unit\meter$\\
    \midrule
    30.000 & 0.000 & 0.030 & 0.030 &   0.533 \\
    50.000 & 0.010 & 0.040 & 0.030 &   1.463 \\
    70.000 & 0.042 & 0.085 & 0.043 &   2.835 \\
    90.000 & 0.040 & 0.160 & 0.120 &   4.633 \\
    110.000 & 0.050 & 0.270 & 0.220 &   6.841 \\
    130.000 & 0.095 & 0.370 & 0.275 &   9.441 \\
    150.000 & 0.130 & 0.500 & 0.370 &  12.420 \\
    170.000 & 0.200 & 0.650 & 0.450 &  15.760 \\
    190.000 & 0.290 & 0.815 & 0.525 &  19.446 \\
    210.000 & 0.310 & 0.950 & 0.640 &  23.461 \\
    230.000 & 0.390 & 1.160 & 0.770 &  27.790 \\
    250.000 & 0.490 & 1.370 & 0.880 &  32.417 \\
    270.000 & 0.580 & 1.560 & 0.980 &  37.325 \\
    290.000 & 0.610 & 1.760 & 1.150 &  42.499 \\
    310.000 & 0.620 & 1.965 & 1.345 &  47.922 \\
    330.000 & 0.730 & 2.180 & 1.450 &  53.579 \\
    350.000 & 0.800 & 2.380 & 1.580 &  59.453 \\
    370.000 & 0.880 & 2.620 & 1.740 &  65.529 \\
    390.000 & 1.000 & 2.880 & 1.880 &  71.791 \\
    410.000 & 1.100 & 3.150 & 2.050 &  78.223 \\
    430.000 & 1.210 & 3.380 & 2.170 &  84.807 \\
    450.000 & 1.300 & 3.640 & 2.340 &  91.530 \\
    470.000 & 1.385 & 3.905 & 2.520 &  98.374 \\
    490.000 & 1.450 & 4.050 & 2.600 & 105.324 \\
    \bottomrule
    \end{tabular}
\end{table}

\begin{figure}[H]
  \centering
  \includegraphics{pictures/Lineare Regression2.pdf}
  \caption{Plot des eckigen, einseitig eingespannten Stabes gegen die lin. Regression.}
  \label{fig:plot2}
\end{figure}


\subsection{Runder Stab, beidseitige Einspannung}

Für die beidseitige Einspannung erhält man die Werte
\begin{align*}
  a &= 0.0036 ± 0.0002 \frac{1}{\unit{\meter\squared}} & \text{und}& & b&= (0.2068 ± 0.0283) \cdot 10^{-3} \unit\meter \text{ .} \\
\end{align*}
Da es sich nun um eine beidseitige Einspannung handelt, muss die Formel (\ref{eq:E2}) angewendet werden.
Somit wird $E$ zu
\begin{equation*}
  E = \frac{m_\text{angehängt} \cdot g}{48 \cdot I \cdot a} = (180 \pm 10) \cdot 10^{9}
\end{equation*}
bestimmt.

%%%%%%%%%%%%%%%%%%%%%%%%%%%%%%%%%%%%%%%%%%%%%%%%%%%
%%%%%%%%%%%%%%%%%%%%%%%%%%%%%%%%%%%%%%%%%%%%%%%%%%%
%%%%%%%%%%%%%%%%%%%%%%%%%%%%%%%%%%%%%%%%%%%%%%%%%%%

%Die Abweichung beträgt somit
%\begin{equation*}
%  \increment E = \left|\frac{180 - 125}{125}\right| \cdot 100 = 44 \% \text{ .}
%\end{equation*}

%%%%%%%%%%%%%%%%%%%%%%%%%%%%%%%%%%%%%%%%%%%%%%%%%%%
%%%%%%%%%%%%%%%%%%%%%%%%%%%%%%%%%%%%%%%%%%%%%%%%%%%
%%%%%%%%%%%%%%%%%%%%%%%%%%%%%%%%%%%%%%%%%%%%%%%%%%%

\begin{table}[H]
  \centering
  \caption{Messwerte der Auslenkung des runden Stabes bei beidseitiger Einspannung.}
  \label{tab:runderstabbeidseitig}
  \begin{tabular}{c c c c c}
    \toprule
    $x / 10 ^{-3} \unit\meter$ &  $D_0 (x) / 10^{-3} \unit\meter$ &
    $D_m (x) / 10^{-3} \unit\meter$ & $D(x) / 10^{-3} \unit\meter$ & $(3L^2x - 4x^3) / 10^{-3} \unit\meter$\\
    \midrule
    30.000 & 0.000 & 0.320 & 0.320 &  31.221 \\
    50.000 & 0.120 & 0.570 & 0.450 &  51.715 \\
    70.000 & 0.240 & 0.770 & 0.530 &  71.729 \\
    90.000 & 0.400 & 0.960 & 0.560 &  91.071 \\
    110.000 & 0.530 & 1.170 & 0.640 & 109.549 \\
    130.000 & 0.640 & 1.360 & 0.720 & 126.971 \\
    150.000 & 0.720 & 1.520 & 0.800 & 143.145 \\
    170.000 & 0.820 & 1.680 & 0.860 & 157.879 \\
    190.000 & 0.910 & 1.810 & 0.900 & 170.981 \\
    210.000 & 0.960 & 1.860 & 0.900 & 182.259 \\
    230.000 & 1.010 & 1.960 & 0.950 & 191.521 \\
    250.000 & 1.060 & 2.030 & 0.970 & 198.575 \\
    270.000 & 1.090 & 2.060 & 0.970 & 203.229 \\
    285.000 & 0.180 & 1.140 & 0.960 & 205.029 \\
    305.000 & 0.200 & 1.130 & 0.930 & 205.021 \\
    325.000 & 0.200 & 1.100 & 0.900 & 202.085 \\
    345.000 & 0.190 & 1.050 & 0.860 & 196.029 \\
    365.000 & 0.180 & 0.980 & 0.800 & 186.661 \\
    385.000 & 0.170 & 0.910 & 0.740 & 173.789 \\
    405.000 & 0.160 & 0.830 & 0.670 & 157.221 \\
    425.000 & 0.150 & 0.740 & 0.590 & 136.765 \\
    445.000 & 0.120 & 0.630 & 0.510 & 112.229 \\
    465.000 & 0.100 & 0.520 & 0.420 &  83.421 \\
    485.000 & 0.100 & 0.460 & 0.360 &  50.149 \\
    505.000 & 0.060 & 0.270 & 0.210 &  12.221 \\
    525.000 & 0.000 & 0.120 & 0.120 & -30.555 \\
    \bottomrule
    \end{tabular}
\end{table}

\begin{figure}[H]
  \centering
  \includegraphics{pictures/Lineare Regression3.pdf}
  \caption{Plot des runden, beidseitig eingespannten Stabes gegen die lin. Regression.}
  \label{fig:plot3}
\end{figure}


\subsection{Eckiger Stab, beidseitige Einspannung}

Es ergeben sich die Werte 
\begin{align*}
  a &= 0.0032 ± 0.0006 \frac{1}{\unit{\meter\squared}} & \text{und}& & b&= (-0.2266 ± 0.0971) \cdot 10^{-3} \unit\meter \text{ .} \\
\end{align*}
Nun kann das Elastizitätsmodul berechnet werden. Es gilt
\begin{equation*}
  E = \frac{m_\text{angehängt} \cdot g}{48 \cdot I \cdot a} = (119 \pm 22) \cdot 10^{9} \text{ .}
\end{equation*}

%%%%%%%%%%%%%%%%%%%%%%%%%%%%%%%%%%%%%%%%%%%%%%%%%%%
%%%%%%%%%%%%%%%%%%%%%%%%%%%%%%%%%%%%%%%%%%%%%%%%%%%
%%%%%%%%%%%%%%%%%%%%%%%%%%%%%%%%%%%%%%%%%%%%%%%%%%%

%Die Abweichung beträgt
%\begin{equation*}
%  \increment E = \left|\frac{119 - 125}{125}\right| \cdot 100 = 4.8 \% \text{ .}
%\end{equation*}

%%%%%%%%%%%%%%%%%%%%%%%%%%%%%%%%%%%%%%%%%%%%%%%%%%%
%%%%%%%%%%%%%%%%%%%%%%%%%%%%%%%%%%%%%%%%%%%%%%%%%%%
%%%%%%%%%%%%%%%%%%%%%%%%%%%%%%%%%%%%%%%%%%%%%%%%%%%

\begin{table}[H]
  \centering
  \caption{Messwerte der Auslenkung des eckigen Stabes bei beidseitiger Einspannung.}
  \label{tab:eckigerstabbeidseitig}
  \begin{tabular}{c c c c c}
    \toprule
    $x / 10 ^{-3} \unit\meter$ &  $D_0 (x) / 10^{-3} \unit\meter$ &
    $D_m (x) / 10^{-3} \unit\meter$ & $D(x) / 10^{-3} \unit\meter$ & $(3L^2x - 4x^3) / 10^{-3} \unit\meter$\\
    \midrule
    30.000 & 0.000 & 0.260 &  -0.260 &  32.508 \\
    50.000 & 0.030 & 0.280 &  -0.250 &  53.861 \\
    70.000 & 0.070 & 0.280 &  -0.210 &  74.733 \\
    90.000 & 0.110 & 0.300 &  -0.190 &  94.933 \\
    110.000 & 0.160 & 0.300 & -0.140 & 114.269 \\
    130.000 & 0.220 & 0.320 & -0.100 & 132.550 \\
    150.000 & 0.270 & 0.270 & -0.000 & 149.582 \\
    170.000 & 0.310 & 0.290 & 0.020 & 165.174 \\
    190.000 & 0.360 & 0.280 & 0.080 & 179.134 \\
    210.000 & 0.440 & 0.210 & 0.230 & 191.271 \\
    230.000 & 0.490 & 0.130 & 0.360 & 201.391 \\
    250.000 & 0.550 & 0.120 & 0.430 & 209.303 \\
    270.000 & 0.630 & 0.040 & 0.590 & 214.815 \\
    295.000 & 0.900 & 1.500 &  0.600 & 218.038 \\
    305.000 & 0.850 & 1.460 &  0.610 & 218.109 \\
    325.000 & 0.800 & 1.400 &  0.600 & 216.031 \\
    345.000 & 0.740 & 1.300 &  0.560 & 210.834 \\
    365.000 & 0.630 & 1.170 &  0.540 & 202.324 \\
    385.000 & 0.550 & 1.050 &  0.500 & 190.310 \\
    405.000 & 0.500 & 0.920 &  0.420 & 174.600 \\
    425.000 & 0.400 & 0.810 &  0.410 & 155.003 \\
    445.000 & 0.340 & 0.680 &  0.340 & 131.325 \\
    465.000 & 0.250 & 0.530 &  0.280 & 103.375 \\
    485.000 & 0.240 & 0.430 &  0.190 &  70.961 \\
    505.000 & 0.110 & 0.300 &  0.190 &  33.892 \\
    525.000 & 0.000 & 0.110 &  0.110 &  -8.026 \\
    \bottomrule
    \end{tabular}
\end{table}

\begin{figure}[H]
  \centering
  \includegraphics{pictures/Lineare Regression4.pdf}
  \caption{Plot des eckigen, beidseitig eingespannten Stabes.}
  \label{fig:plot4}
\end{figure}