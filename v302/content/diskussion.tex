\section{Diskussion}
\label{sec:Diskussion}

Bei allen Widerstandsbestimmungen wurde der Fehler nur durch die Bauteile
bestimmt. Dies lässt im Prinzip auf eine sehr gute Messung schließen, die ohne
neue Bauteile nicht weiter verbessert werden kann. 

Bei dem Vergleich der Messung der Induktivität und des Innenwiderstandes der Spule
in \autoref{sec:induktausw} und \autoref{sec:maxwausw} fallen zwei Unterschiede auf. 
So sind die Fehler bei der Messung mit der Maxwell-Brücke deutlich kleiner, was erneut auf eine genauere
Messmethode schließen lässt. 

Die in \autoref{sec:wienausw} gemessenen Daten besitzen jedoch eine sehr hohe Abweichung 
von der Theoriekurve, obwohl der Klirrfaktor relativ gering erscheint.
Dies liegt daran, dass dieser nur ein Maß der Qualität des Sinusspannungsgenerators im Bereich des 
Spannungsminimums darstellt. 
Die hohe Abweichung von der Theorie außerhalb des Minimums lässt sich also zuminndest auf 
bautechnischen Probleme schließen, wie etwa fehlerhafte Potentiometer oder Induktivitäten.
