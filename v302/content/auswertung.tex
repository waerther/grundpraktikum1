\section{Auswertung}
\label{sec:Auswertung}

\subsection{Wheatstonesche Brücke} \label{sec:wheatausw}

Die Werte der bei der Messung bekannten Widerstände sind in \autoref{tab:wheatbekannt} aufgetragen.
\begin{table}
  \centering
  \caption{Werte der bekannten Widerstände.}
  \label{tab:wheatbekannt}
  \begin{tabular}{lcc}
    \toprule
     & Messung 1 & Messung 2  \\
    \midrule
    \multicolumn{3}{c}{ Bestimmung von $R_{\text{x, 13}}$ } \\
    $R_2 \mathbin{/} \unit{\ohm}$ & 1000 & 500 \\
    $R_3 \mathbin{/} \unit{\ohm}$ &  240 & 388 \\
    $R_4 \mathbin{/} \unit{\ohm}$ &  760 & 612 \\
    \midrule 
    \multicolumn{3}{c}{ Bestimmung von $R_{\text{x. 18}}$ } \\
    $R_2 \mathbin{/} \unit{\ohm}$ & 1000 & 500 \\
    $R_3 \mathbin{/} \unit{\ohm}$ &  190 & 321 \\
    $R_4 \mathbin{/} \unit{\ohm}$ &  810 & 679 \\
    \bottomrule
  \end{tabular}
\end{table}

Die baubedingte Abweichung für $R_2$ beträgt $\qty{0.2}{\percent}$. 
Die Abweichung für das Verhältnis $\frac{R_3}{R_4}$ beträgt $\qty{0.5}{\percent}$.
Mit den bekannten Werten $R_2$, $R_3$ und $R_4$ und der (\ref{eq:Rx}) ergibt sich
mithilfe der Fehlerrechnung nach \autoref{sec:Fehlerrechnung} für die gemittelten unbekannten Widerstände
\begin{align*}
  \bar{R}_\text{x, 13} &= \qty{316.4(1.2)}{\ohm} \quad \text{und} \\
  \bar{R}_\text{x, 18} &= \qty{235.5(9)}{\ohm} \, .
\end{align*}


\subsection{Kapazitätsmessbrücke} \label{sec:kapazausw}

Die Werte der bekannten Kenngrößen sind in \autoref{tab:kapazbekannt} zu sehen.
\begin{table}
  \centering
  \caption{Werte der bekannten Kapazitäten und Widerstände.}
  \label{tab:kapazbekannt}
  \begin{tabular}{cc}
    \toprule
     & Messung 1  \\
    \midrule
    \multicolumn{2}{c}{ Bestimmung von $C_{\text{x, 8}}$ und $R_{\text{x, 8}}$ } \\
    $C_2 \mathbin{/} \unit{\nano\farad}$     & 450 \\
    $R_2 \mathbin{/} \unit{\ohm}$            & 500 \\
    $R_3 \mathbin{/} \unit{\ohm}$            & 388 \\
    $R_4 \mathbin{/} \unit{\ohm}$            & 612 \\
    \bottomrule
  \end{tabular}
\end{table}

Die Abweichung für $C_2$ beträgt $\qty{0.2}{\percent}$, wobei die Abweichung des Potentiometers beibleibt
und die Abweichung für $R_2$ ist nun $\qty{3}{\percent}$. 
Die Berechnung von $C_\text{c}$ und $R_\text{c}$ geschieht nach (\ref{eq:Rx}):
\begin{align*}
  C_\text{x, 8} &= \qty{233.7(1.3)}{\nano\farad} \, , \\
  R_\text{x, 8} &= \qty{433.0(13.0)}{\ohm} \, .
\end{align*}


\subsection{Induktivitätsmessbrücke} \label{sec:induktausw}

Die Werte der bekannten Kenngrößen sind in \autoref{tab:unduktbekannt} zu sehen.
\begin{table}
  \centering
  \caption{Werte der bekannten Induktivität und der Widerstände.}
  \label{tab:unduktbekannt}
  \begin{tabular}{lccc}
    \toprule
     & Messung 1 & Messung 2 & Messung 3  \\
    \midrule
    \multicolumn{4}{c}{ Bestimmung von $L_{\text{x, 17}}$ und $R_{\text{x, 17}}$ } \\
    $L_2 \mathbin{/} \unit{\milli\henry}$ &  14,6 &  27,5 &  20,1\\
    $R_2 \mathbin{/} \unit{\ohm}$         &  30,0 &  33,0 &  42,0\\
    $R_3 \mathbin{/} \unit{\ohm}$         & 745,0 & 606,0 & 678,0\\
    $R_4 \mathbin{/} \unit{\ohm}$         & 255,0 & 396,0 & 322,0\\
    \bottomrule
  \end{tabular}
\end{table}

Die baubedingten Abweichungen sind dieselben wie im vorangegangenem \autoref{sec:kapazausw}.
Die Abweichung für die Induktivität $L_2$ der Spule beträgt $\qty{0.2}{\percent}$.
Die gemittelten Werte für $L_\text{x, 17}$ und $R_\text{x, 17}$ ergeben sich nach (\ref{eq:L_x}) und (\ref{eq:induktR_x}) zu
\begin{align*}
  \bar{L}_\text{x, 17} &= \qty{42.4(1)}{\milli\henry} \quad \text{und} \\
  \bar{R}_\text{x, 17} &= \qty{75.5(1.4)}{\ohm} \, .
\end{align*}

\subsection{Maxwell Brücke} \label{sec:maxwausw}

Die Werte der bekannten Kenngrößen sind in \autoref{tab:maxwbekannt} ersichtlich.
\begin{table}
  \centering
  \caption{Werte der bekannten Kapazität und der Widerstände.}
  \label{tab:maxwbekannt}
  \begin{tabular}{lcc}
    \toprule
     & Messung 1 & Messung 2  \\
    \midrule
    \multicolumn{3}{c}{ Bestimmung von $L_{\text{x, 17}}$ und $R_{\text{x, 17}}$ } \\
    $C_4 \mathbin{/} \unit{\nano\farad}$  &  450 &  660 \\
    $R_2 \mathbin{/} \unit{\ohm}$         & 1000 & 1000 \\
    $R_3 \mathbin{/} \unit{\ohm}$         &   85 &   60 \\
    $R_4 \mathbin{/} \unit{\ohm}$         &  960 &  700 \\
    \bottomrule
  \end{tabular}
\end{table}

Die Abweichungen der beiden einstellbaren Widerstände $R_3$ und $R_4$ betragen jeweils  $\qty{3}{\percent}$.
Die des Widerständs $R_2$ und der Kapazität $c_4$ des Kondensators betragen beide $\qty{0.2}{\percent}$.
Die gemittelten Werte für $L_\text{x, 17}$ und $R_\text{x, 17}$ ergeben sich nach (\ref{eq:maxwL_x}) und (\ref{eq:Rx}) zu
\begin{align*}
  \bar{L}_\text{x, 17} &= \qty{47.2(1.4)}{\milli\henry} \quad \text{und} \\
  \bar{R}_\text{x, 17} &= \qty{105.0(4.0)}{\ohm} \, .
\end{align*}


\subsection{Wien-Robinson-Messbrücke} \label{sec:wienausw}

In dieser Messreihe wird die Frequenzabhängigkeit einer Wien-Robinson-Messbrücke untersucht.
Hierzu wird das Verhältnis der Brückenspannung $U_\text{Br}$ zur Speisespannung $U_\text{S}$
gegen $\Omega = \frac{\nu}{\nu_0}$ abgetragen. Wobei sich die Frequenz $\nu_0$, 
bei der die Brückenspannung verschwinden sollte, sich ergibt durch:
\begin{align*}
  \omega_{0} &=\frac{1}{R C}=\frac{1}{\qty{1}{\kilo\ohm} \cdot \qty{660}{nF}}= \qty{1515.15}{Hz} \\
  \Leftrightarrow \nu_{0} &=\frac{\omega_{0}}{2 \pi} = \qty{241.14}{Hz} \, .
\end{align*}
Das Minimum wird, im Vergleich dazu sehr genau, bei $\nu_\text{0, exp} = \qty{241}{Hz}$ gemessen.

\begin{table}
  \centering
  \caption{Gemessene Spannungen in Abhängigkeit von der Frequenz am Sinusspannungsgenerator.}
  \label{tab:spannung}
  \begin{tabular}{c c c}
    \toprule
    $\nu \mathbin{/} \mathrm{Hz}$ &  $U_\text{Br} \mathbin{/} \mathrm{mV}$ & $\frac{U_\text{Br}}{U_\text{s}}$ \\
    \midrule
        20 &     280 & 0,068 \\
        40 &     260 & 0,063 \\
        80 &     200 & 0,049 \\
       160 &      80 & 0,020 \\
       230 &      32 & 0,008 \\
       232 &      26 & 0,006 \\
       234 &      20 & 0,005 \\
       236 &      15 & 0,004 \\
       238 &      10 & 0,002 \\
       240 &       4 & 0,001 \\
       241 &       2 & 0,001 \\
       242 &       5 & 0,001 \\
       244 &      10 & 0,002 \\
       246 &      15 & 0,004 \\
       248 &      20 & 0,005 \\
       250 &      26 & 0,006 \\
       320 &      55 & 0,013 \\
       640 &     180 & 0,044 \\
      1280 &     245 & 0,060 \\
      2560 &     280 & 0,068 \\
      5020 &    1000 & 0,244 \\
     10040 &    1400 & 0,341 \\
     20080 &    1000 & 0,244 \\
     30000 &     750 & 0,183 \\  
    \bottomrule
    \end{tabular}
\end{table}

Die dabei gemessene Speisespannung erweist sich zu jeder Frequenz als konstant: $U_\text{S} = \qty{4}{V}$.
In \autoref{fig:plot} sind die Messdaten und die Theoriekurve nach (\ref{eq:Wien2}) dargestellt.
\begin{figure}
  \centering
  \includegraphics[width=\textwidth]{plot.pdf}
  \caption{Vergleich von Messdaten mit der Theoriekurve.}
  \label{fig:plot}
\end{figure}

Für die Bestimmung des Klirrfaktors wird zunächst genähert, dass die Summe der
Oberwellen nur von dem Term der zweiten Oberwelle bestimmt wird. Dabei ist es egal,
ob die Effektivspannung oder die Amplituden der Spannung betrachtet werden. Im Fol-
gendem wird mit der Effektivspannung gerechnet, da diese bereits berechnet wurde.
Nach (\ref{eq:klirr}) werden die Werte für $U_2$ und $U_1$ benötigt, wobei $U_1 = \qty{4}{\volt}$ und
von $U_\text{S} = \qty{0.002}{\milli\volt}$ bei $\nu_0$ ist. 
Mithilfe von (\ref{eq:Wien2}) folgt für
\begin{align*}
  U_{2} &=\frac{0,002 \mathrm{~V}}{\sqrt{\frac{\left.\left(2^{2}-1\right)^{2}\right)}{9 \cdot\left[\left(1-2^{2}\right)^{2}+9 \cdot 2^{2}\right]}}} \\
  &= \qty{13,041}{\milli\volt} \, . \\
  \intertext{Der Klirrfaktor ergibt sich entprechend zu}
  k&=\frac{U_{2}}{U_{1}}= 3,35 \cdot 10^{-3} \, .
\end{align*}