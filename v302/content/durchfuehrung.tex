\section{Durchführung}
\label{sec:Durchführung}

Zunächst werden die Geräte eingestellt.
Die Anregungsfrequenz wird auf $1 \si{\kilo\hertz}$ eingestellt.
Die maximale Spannung darf $1 \si{\volt}$ nicht überschreiten.

\subsection{Wheatstonesche Brücke} \label{sec:wheatdurchf}
Es erfolgt ein Aufbau nach dem Beispiel in Abbildung \ref{fig:Schaltung2}.
Dann wird das Potentiometer so eingestellt, bis die Brückenspannung minimal wird.
Im Idealfall verschwindet diese komplett.
Die drei bekannten Widerständer werden notiert.
Als unbekannter Widerstand wird \enquote{Wert 13} benutzt.
Ist die Messung abgeschlossen, wird diese nochmal für einen anderen unbekannten Widerstand durchgeführt.
Hier wird \enquote{Wert 18} in den Schaltkreis eingebaut.

\subsection{Kapazitätsmessbrücke} \label{sec:kapazdurchf}

Es wird die Schaltung in Abbildung \ref{fig:Schaltung2} nachgebaut.
Als unbekannter Kondensator wird \enquote{Wert 8} gewählt.
Nun wird ebenfalls das Potentiometer solange justiert, bis die Brückenspannung minimal ist oder verschwindet.
Es werden wieder alle Werte notiert, auch die Werte der bekannten Widerstände.

\subsection{Induktivitätsmessbrücke} \label{sec:induktdurchf}

Mit der Induktivitätsmessbrücke wird die Induktivität $L$ und der ohmsche Widerstand $R$ der Spule bestimmt.
Auch hier widr das Potentiometer solange justiert, bis die Brückenspannung minimal ist.
Im Optimalfall wird diese ebenfalls komplett verschwinden.
Es werden die Werte der bekannten Widerstände aufgenommen.
Als unbekannten Wert wird \enquote{Wert 17} benutzt.

\subsection{Maxwell Brücke} 

Nun soll mit der Maxwellbrücke die selbe Spule bestimmt werden, wie mit der Induktivitätsmessbrücke.
Dafür wird eine Schaltung gemäß \ref{fig:Schaltung5} aufgebaut.
Diesesmal werden die einzelnen Widerstände variiert, bis auch hier die 
Brückenspannung minimal wird.
Es werden wieder alle Bekannten und Unbekannten notiert.

\subsection{Wien-Robinson-Messbrücke}

Es wird die Schaltung aus Abbildung \ref{fig:Schaltung6} nachgebaut.
Es wird im Bereich $\nu \in \left[ 30 ;30000 \right] \si\hertz $ gemessen.
Dabei wird um das Minimum die Schrittweise verringert, um die Genauigkeit zu verbessern.
Ansonsten wird die Spannung immer verdoppelt.
Es werden also Frequenz $\nu$, die Brückenspannung $U_\text{Brücke}$ und die Speisespannung $U_\text{S}$
bei jedem Schritt notiert.