\section{Durchführung}
\label{sec:Durchführung}

Zunächst werden die Geräte eingestellt.
Die Anregungsfrequenz wird auf $1 \si{\kilo\hertz}$ eingestellt.
Die maximale Spannung darf $1 \si{\volt}$ nicht überschreiten.

\subsection{Wheatstonesche Brücke} \label{sec:wheatdurchf}
Es erfolgt ein Aufbau nach dem Beispiel in Abbildung \ref{fig:Schaltung2}.
Dann wird das Potentiometer so eingestellt, bis die Brückenspannung minimal wird.
Im Idealfall verschwindet diese komplett.
Die drei bekannten Widerständer werden notiert.
Ist die Messung abgeschlossen, wird diese nochmal für einen anderen unbekannten Widerstand durchgeführt.

\subsection{Induktivitätsmessbrücke} \label{sec:induktdurchf}