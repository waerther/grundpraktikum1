\section{Theorie}
\label{sec:Theorie}

Mit der Phase eines Stoffes wird ein räumlich abgegrenzter Bereich in einem abgeschlossenen System beschrieben,
in dem sich der Stoff in einem physikalisch homogenen Zustand befindet. 
Darunter gelten unter anderem die Aggregatzustände: fest, flüssig und gasförmig.
In einem sogenannten Phasendiagramm (siehe Abbildung \ref{fig:phasendiagramm}) besitzt ein System
innerhalb eines abgegrenzten Bereichs zwei Freiheitsgrade, den Druck $p$ und die Temperatur $T$.
Das heißt, dass diese ohne Phasenänderung variiert werden können, solange keine Grenzlinie überschritten wird.

\begin{figure} 
    \centering
    \includegraphics[width=12cm] {pictures/phasendiagramm.pdf}  
    \caption{Qualitatives Phasendiagramm von Wasser. \cite[1]{v203}}
    \label{fig:phasendiagramm}
\end{figure} 