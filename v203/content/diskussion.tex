\section{Diskussion}
\label{sec:Diskussion}

Im Vergleich zum Literaturwert \cite{enthalpy} fällt auf, 
dass der berechnete Wert für die Verdampfungswärme $L$ nur circa halb so groß ist. 
Ursache dafür ist unter anderem, dass die Verdampfungswärme als konstant angenommen wurde. 
Um dem entgegen zu wirken, wurde die Kühlung stetig verringert. 
Da dies allerdings manuell geschah, 
kann nicht sichergestellt werden, dass die Kühlung in einer passenden Rate herunter gedreht wurde. 
Des Weiteren war die Wasserstrahlpumpe nicht stark genug, um die Apparatur gut genug zu evakuieren. 
Schließlich ist es erwähnenswert, dass die Apparatur zu schnell erhitzt wurde, 
sodass das Messgerät sich nicht schnell genug einstellen konnte und Messung die Messung erschwerte.

Bei der Überprüfung der Temperaturabhängigkeit der Verdampfungswärme ist vorerst festzustellen,
dass nur der erste Fall, in Abbildung \ref{fig:plot4} dargestellt, physikalisch sinnvoll ist.
Dies liegt daran, dass sich die Verdampfungswärme bei steigender Temperatur verringern muss, 
damit sie am kritischen Punkt null wird, um zu ermöglichen, dass die gasförmige
und flüssige Phase koexistieren können. 
Auch fällt der sehr hohe Anfagswert von $L$ auf, dessen Fehler vermutlich durch das 
anfängliche einstellen des Gleichgewichtszustands erklärt werden kann.
Ab dem Temperaturwert $\qty{420}{K}$ folgen die Daten in etwa dem Trend der Literaturdaten \cite{enthalpy}.